The main conclusions from this paper come from defining and implementing an ESG factor.  In the first set of analyses where the ESG term is defined as the difference between the MSCI ESG index and the petroleum index, I observed a statistically significant result for the ESG term, which matched previous work conducted on ESG factor modeling. However, in the second set of analyses, when developing an ESG factor from the MSCI ratings, I did not find a statistically significant factor on the aggregate level. Examining the MSCI results at the one-digit SIC level, I was only able to find the firms starting with SIC code  1  to demonstrate a positive and statistically significant factor. These findings reaffirm that the petroleum industry may have negative expected returns but does not build a case for ESG providing excess returns for other industries. This lack of uniformity across both definitions fosters a number of important questions to address in the implementation of the Fama french six-factor model. This uncertainty in the findings of an ESG factor parody the literature review and make it extremely difficult to recommend a pro ESG policy proposal. 

Policy for EGS investing over the last year has been a contentious topic of debate. In 2020, the Department of Labor ruled that private pension plan providers ``cannot invest in ESG vehicles that sacrificed investment returns or take on additional risk" \cite{beals_trump_nodate}. Investment and retirement planning are essential to the wellbeing of everyone in their retirement, and 401(k) plans are a key part of securing the ability to continue a comfortable life into retirement.  Prioritizing returns over a longer period,  even by a few basis points can have a sizable impact when considering the effect of compounding over a time frame of 40 years. When comparing the current policy position to the results from this study, it would seem like a clear solution to uphold the policy laid out by the Department of Labor. However, the results suggest that investors would be better off if pension providers actively avoided traditional fossil fuels as the core of energy production and utilization transitions to renewable solutions. 

Further, it is important to note that the results from this study are retrospective and do not account for future development. Under the Biden administration, there has been a shift in political tone regarding environmental and social concerns that would support corporate ESG initiatives \cite{newburger_heres_2021}. As such, many industries might benefit from these new political tailwinds and create industry ESG divergence similarly to those observed in the petroleum industry. 

To account for this uncertainty and opportunity,  I would recommend rewording the mandate from the Bureau of Labor to be more lenient and offer investment solutions that incorporate ESG for those customers who have an active interest in financing global environmental and social reform. However,  these investment products should also be provided with additional warnings for the inventor and strict regulations regarding performance claims. Return performance should demonstrate the excess return or loss associated with the product's historical performance compared to a not ESG counterpart fund. 

Looking beyond this paper's scope to future research opportunities, there are a few natural extensions for assessing the role rating agencies and indexes play in the EGS landscape. One extension is to further isolate ESG premia by conducting an event study methodology on several well-known ESG indexes, including the MSCI ESG leaders index referenced in this paper. Using addition and deletion events from this index can provide a tangible event to test the market reaction to changes in ESG news and assess if there are abnormal returns. This methodology for evaluating abnormal returns can also be applied to ESG rating changes; however, there might be less of an effect if the market was anticipating a change in the rating similar to corporate credit ratings. As such, the indexing market event might provide a more precise image regarding market tracking behavior around ESG.

Another possible extension for ESG research is to take a further look at the rating agencies. Over the last few years, more companies have entered into the ESG rating space, MSCI and Sustainalytics have held a dominant position in the market, but S\&P, Moody's, and other firms are looking to expand their research efforts around ESG research. An interesting extension would be to analyze the risk premia for different firms and see if there is an ESG factor where high ESG minus low ESG has a larger risk premium for one rating agency over another rating agency.