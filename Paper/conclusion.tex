The main conclusions from this paper comes from defining and  implementing an ESG factor.  In the first set of analysis where ESG term is defined as the different between the MSCI ESG index and petroleum index I observed a statistically significant result for the ESG term which matched previous work which has been conducted on ESG factor modeling. However in the second set of analysis when developing an ESG factor from the MSCI ratings I was unable to find a statistically significant factor on the aggregate level and I was only able to find a statistically significant factor for \hl{ incert number of industries}. This lack of uniformity across both definitions fosters a number of important questions to address in the implementation of FFFF models

Looking beyond this paper's scope and future research opportunities, there are a few natural extensions for assessing the role rating agencies and indexes play in the EGS landscape. One extension is to further isolate ESG premia by conducting an event study methodology on several well-known ESG indexes, including the MSCI ESG leaders index referenced in this paper. Using addition and deletion events from this index can provide a tangible event to test the market reaction to changes in ESG news and assess if there are abnormal returns. This methodology for evaluating abnormal returns can also be applied to ESG rating; however, there might be less of an effect if the market was anticipating a change in the rating. As such, the indexing market event might provide a more precise image regarding market tracking behavior around ESG.

Another possible extension for ESG research is to take a further look at the rating agencies. Over the last few years, more companies have entered into the ESG rating space, MSCI and Sustainalytics have held a dominant position in the market, but S\&P, Moody's, and other firms are looking to expand their research efforts around ESG research. An interesting extension would be to analyze the risk premia for different firms and see if there an ESG factor where high ESG minus low ESG has a larger risk premium for one rating agency over another rating agency.
