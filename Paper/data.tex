The data used for this paper was sourced from four datasets. The daily performance of the individual stock prices was collected from CRSP's primarily daily stock prices. Data for the daily returns of MSCI World ESG Focus Index and S\&P petroleum index was sourced from CAPIQ Compustat. Data on firm-level ESG scores (by ticker) was sourced from MSCI and reflected the most recent company ratings as MSCI does not have the historical ratings relatively available. Finally, daily Fama French factors for SMB, CMA, HML, and RMW were sourced from the Kenneth R. French - Data Library.  


The data and methodology for the paper are broken into two sections; first, a preliminary study is conducted using a broad base ESG factor to confirm a statistically risk sensitivity relationship between ESG and returns. Next, an analysis on a SIC level is conducted in an attempt to understand how industry-specific ESG factors affect the risk-return relationship of stock performance. 

In the first stage of the analysis, an ESG factor is created by taking two market indexes and subtracting their daily returns in an attempt to isolate ESG based risk. The formula for this factor is represented in Equation \eqref{making_EMP}, and for the remainder of the section, one analysis will be referred to as EMP, which is sort for ESG minus petroleum. This method of creating a factor is very similar to Fama and French in their other four factors in the five-factor model. For this analysis, I will be using the MSCI World ESG Focus Index as the high ESG portfolio and the  S\&P petroleum index as the low ESG portfolio. The methodology for creating an ESG factor is similar to the work from \hl{put the author's name in here}. In \hl{put the author's name in here}'s work, they were able to find a statistical risk relationship between a high ESG portfolio and a petroleum index\cite{Frynas2005TheFD}. The petroleum index provides a good contra ESG portfolio as the petroleum industry is in direct opposition to environmental goals that are E components of ESG. A similar analysis could also be run with other market indexes. The S\&P Mining index is another popular index that also performs very poorly for environmental and social concerns\cite{Kapelus2002MiningCS}. A final index that also could represent a contra ESG portfolio is the  \$VICE ETF. This financial product tracks the market for  alcohol, drug, gaming, and betting industry which have all been shown as clear contra ESG industries \hl{inset citation}
\begin{equation}
    \label{making_EMP}
    EMP = R_{\text{World ESG Focus Index}}- R_{\text{S\&P petroleum index}}
\end{equation}



To conduct this analysis, daily data was collected between June 2011 and March 2021.  June 2011 marks the initiation of the MSCI ESG index; as such, this is the beginning of the sample period for this paper. Daily individual stock return information was collected from CRSP and merged with the daily ESG factor returns. The return data was collapsed on month to create a monthly return figure by summing all of the daily returns for each given month in the dataset. Next, the Fama French monthly factors were merged into the dataset for each individual firm for each month. After aggregating, the factor and individual firm return, there were 830,891 observations of monthly return data for 12,194 unique stocks. The merged and cleaned dataset had an average of 68 months of data for each individual security. Table 1 provides brief summary statistics for the data in the first stage of the analysis.

\begin{center}
    \paperspacingnarrow
    {
\def\sym#1{\ifmmode^{#1}\else\(^{#1}\)\fi}
\begin{tabular}{l*{1}{cccc}}
\hline\hline
                    &        mean&          sd&         min&         max\\
\hline
Excess Returns      &       0.008&       0.152&      -3.701&      14.541\\
Market              &       0.009&       0.042&      -0.146&       0.131\\
HML                 &      -0.004&       0.027&      -0.176&       0.094\\
SMB                 &      -0.001&       0.026&      -0.160&       0.077\\
RMW                 &       0.000&       0.015&      -0.043&       0.049\\
CMA                 &      -0.001&       0.014&      -0.036&       0.045\\
EMP                 &      -0.007&       0.090&      -0.495&       0.361\\
Market Cap          & 4299884.523&23004970.838&      82.632&   2.164e+09\\
\hline
Observations        &      830889&            &            &            \\
\hline\hline
\end{tabular}
}
\\
    \textbf{\hl{Table 1: EXPLAIN THE TABLE!!!}}
    \paperspacingwide
\end{center}

In the second stage of the analysis, I look at how SIC code-specific ESG factors affect the risk-return relationship. In this study, I used the MSCI ESG rating by ticker to created SIC-based ESG factors. MSCI rate companies on their ESG score AAA-CCC where AAA is the highest ESG rating and CCC is the lowest ESG rating. In the sample of MSCI ratings, 16 companies were AAA, 88 companies were AA, 118 companies were A, 170 companies were BBB, 108 companies were BB, 40 companies were B, and 5 companies earned a CCC rating.  Each of these ratings was assigned a mating numeric score (1-7) and were merged using the CRSP name file to their corresponding kypermno value by matching ticker information. Subsequently, the ESG scores were merged into the monthly return dataset described at the end of the first analysis. A final process was conducted to make a unique ESG factor for each of the SIC codes. The unique ESG factor was created at the first digit SIC level and generated by splitting the SIC level ESG data into deciles and taking the top 10\% and as the high ESG index bottom 10\% as the low ESG and subtracting high ESG minus low ESG as explained in equation \eqref{making_EMP_section}. This ESG factor was created for SIC codes beginning with 1-4 and 6-8. SIC code. SIC code 5 only had BBB ratings, and MSCI did not have significant data on SIC code 0 or 9. In addition to creating industry specific ESG factors I also created a all industry ESG factor using the top 10\% and bottom 10\% for all MSCI ESG scores to create an ESG factor. \hl{explain how EMP is the same down here for consistency of reuslts}

\begin{equation}
    \label{making_EMP_section}
    EMP_i = R_{\text{high ESG,i}}- R_{\text{low ESG,i}}
\end{equation}

\begin{center}
    Where i is the first digit SIC code    
\end{center}



