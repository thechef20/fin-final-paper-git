The data used for this paper was sourced from four datasets. The daily performance of the individual stock prices was collected from CRSP's primarily daily stock prices. Data for the daily returns of MSCI World ESG Focus Index and S\&P petroleum index was sourced from CAPIQ Compustat. Data on firm-level ESG scores (by ticker) was sourced from MSCI and reflected the most recent company ratings. Finally, monthly Fama French factors for SMB, CMA, HML, and RMW were sourced from the \textit{Kenneth R. French Data Library}.  


The data and methodology for the paper are broken into two sections; first, a preliminary study is conducted using a broad base ESG factor constructed from two market indexes to confirm a statistical relationship between ESG and excess returns. Next, another ESG factor is constructed using the MSCI ESG dataset  to isolate additional ESG risk.

In the first stage of the analysis, an ESG factor is created by taking two market indexes and subtracting their daily returns in an attempt to isolate ESG based risk. The formula for this factor is represented in Equation \eqref{making_EMP}, and for the remainder of this analysis, this ESG factor will be referred to as the EMP factor (ESG minus Petroleum). This method of creating a factor is based on the fama french model used by the other four factors in the five-factor model. For this analysis, I will be using the MSCI World ESG Focus Index as the high ESG portfolio and the  S\&P petroleum index as the low ESG portfolio. The choice of the petroleum index was based on Gregory et al's work where the authors demonstrated the sensitivity of the index to contra ESG factors \cite{Frynas2005TheFD}. The petroleum index provides a good contra ESG portfolio as the petroleum industry is in direct opposition to environmental goals that are `E' components of ESG. A similar analysis could also be run with other market indexes. The S\&P Mining index is another popular index that also performs very poorly for environmental and social concerns\cite{Kapelus2002MiningCS}. A third index that also could represent a contra ESG portfolio is the  \$VICE ETF. This financial product tracks the market for alcohol, drug, gaming, and betting industry which have all been shown as clear contra ESG industries \cite{Grougiou2016CorporateSR}.

\begin{equation}
    \label{making_EMP}
    EMP = R_{\text{World ESG Focus Index}}- R_{\text{S\&P petroleum index}}
\end{equation}


To conduct this analysis, daily data was collected between June 2011 and March 2021.  June 2011 marks the initiation of the MSCI ESG index; as such, this is the beginning of the sample period for this paper. Daily individual stock return information from CRSP was merged with the daily ESG factor returns. The resulting merged data was collapsed on months to create a monthly return figure by summing all of the daily returns for each given month in the dataset. Next, the Fama French monthly factors were merged onto the dataset for each individual firm for each month. After aggregating, the factor and individual firm return, there were 830,889 observations of monthly return data for 12,194 unique stocks. The merged and cleaned dataset had an average of 68 months of data for each individual security. Table 1 provides brief summary statistics for the data in the first stage of the analysis.

\begin{center}
    \paperspacingnarrow
    {
\def\sym#1{\ifmmode^{#1}\else\(^{#1}\)\fi}
\begin{tabular}{l*{1}{cccc}}
\hline\hline
                    &        mean&          sd&         min&         max\\
\hline
Excess Returns      &    .0080939&    .1519073&   -3.700782&    14.54115\\
Market              &     .008514&    .0420862&   -.1456952&    .1308475\\
HML                 &   -.0044938&    .0274506&      -.1758&       .0942\\
SMB                 &   -.0011178&    .0256134&      -.1598&       .0771\\
RMW                 &    .0004993&    .0152985&      -.0435&       .0494\\
CMA                 &   -.0012623&    .0144457&      -.0364&       .0454\\
EMP                 &   -.0072447&    .0904227&   -.4945351&    .3614013\\
Market Cap          &     4299885&    2.30e+07&     82.6317&    2.16e+09\\
\hline
Observations        &      830889&            &            &            \\
\hline\hline
\end{tabular}
}
\\
    \textbf{Table 1:} shows summary statistics for the data set
    \paperspacingwide
\end{center}

In the second stage of the analysis, the definition of the ESG factor is changed to isolate additional ESG risk. In the first part of the analysis, the contra ESG portfolio's main concern centered on environmental risk. To increase exposure to social and governance while maintaining exposure to environmental risk, a portfolio is constructed from MSCI ESG ratings. MSCI provides public ratings for 546 firms and issues a rating from AAA-CCC where AAA is the highest ESG rating, and CCC is the lowest ESG rating. In the sample of MSCI ratings, 16 companies were AAA, 88 companies were AA, 118 companies were A, 170 companies were BBB, 108 companies were BB, 40 companies were B, and 5 companies earned a CCC rating.  Each of these ratings was assigned a matching numeric score (1-7) and was merged using the CRSP name file to their corresponding kypermno value by matching ticker information. The data was merged onto the monthly return dataset described at the end of the first analysis. To create an ESG factor with the MSCI rating, numeric ratings were split into deciles, and a high minus low  ESG factor was generated by taking the top 10\% of ESG firms and subtracting that from the bottom 10\%  of ESG firms as described in equation \eqref{making_EMP_section}. For the remainder of this analysis, the MSCI ESG factor will be referred to as the ESG factor or the MSCI ESG factor. 

\begin{equation}
    \label{making_EMP_section}
    EMP = R_{\text{top 10\% ESG Firms}}- R_{\text{Bottom 10\% ESG Firms}}
\end{equation}

