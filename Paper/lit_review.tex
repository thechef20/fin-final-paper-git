The modern foundation of asset pricing began with the capital asset pricing model based on Sharp's 1964 work which used a regression analysis to relate market returns to individual security returns \cite{Sharpe1964CAPITALAP}. The CAPM model was later iterated on by Fama french in 1993 with the 3 factors, which considered size,  book, and the market terms from the CAPM model. Fama French found that these three coefficients created a more explanatory model which demonstrated the relationship that value companies (low book to market) and small companies experience larger excess returns \cite{Fama1992TheCO}. In addition to adding the two factors, Fama French also incorporated the two-pass sort methodology, which isolates portfolio sensitivity and risk premia in a multi sep regression process \cite{Kleibergen2013UnexplainedFA, Fama1973RiskRA}.

Fama and French innovated on the three-factor model in 2015 by adding two additional regressors, investment and profitability. The five-factor model analysis found that the addition of the investment and profitability factor made the value factor redundant \cite{Fama2013AFA}. The reliability of the five-factor has been shown by replications on international exchanges and has been found to hold true in Germany, China, and Japan but did not hold true for Australia \cite{Fama2015InternationalTO, LIN2017141, Chiah2016A}. The Fama French five factor model is the standard in asset pricing and is the basis for many academic research projects that extend the five-factor model to address additional unpriced factors \cite{Dirkx2020TheFF}. Gregory and Stead use a sixth factor as an ESG factor to analyze if the ESG effect played a significant role in determining risk premia. This paper found that  ``the risk premium of the sustainability factor is positive and significant"\cite{Gregory2020TheGP}.


Literature on ESG investing has become a quickly growing academic interest, with researchers finding both strong evidence for a non-priced ESG factor and no evidence of a noticeable ESG factor. In addition to Gregory and Stead's paper, several other authors have used the Fama French model to address the question concerning ESG and risk premia. To set up an ESG Fama French model, many academic journal articles use an index as a positive ESG portfolio and another index as a negative portfolio to create the ESG factor. In Gregory and Stead's paper, the author uses the S\&P ESG index as the high ESG index and the S\&P petroleum index as the low index to create a factor by subtracting the returns from the high ESG portfolio from the low ESG portfolio \cite{Gregory2020TheGP}. In Dorfleitner et al.'s paper, the authors collect Thomas Reuter's controversies and combined ESG score for 2500 companies. With the ranked ESG score data, the author then creates two portfolios by separating the top decile score and the bottom deciles scores, which act as the ESG and contra ESG portfolios. The author finds that there are measurable excess returns and contributes. The main findings for this paper is the realization that small companies, particularly those who stay out of controversy, trend to be undervalued and appreciate compared to similar size companies with lots of controversy \cite{Dorfleitner2020ESGCA}.

Although authors have found positive evidence, other authors have found no such evidence of a relationship. Work completed by Liou investigates the MSCI ESG rating data to construct equal-weighted portfolios as far back as 1991  and performs a Fama French analysis. In the paper, the author finds that in a post-GFC test, there is no excess return for a long strong ESG short poor ESG factor. However, the author does note that there is a relationship to the size and the coefficient for the ESG factor \cite{Lioui2018ESGFI}. 

Additionally, some authors have looked at the excess returns on ESG portfolios and assess the effects over calendar years. Work from Bennani et al. explored the post-GFC landscape for ESG investing and found that in North American stocks from 2010-2013, investing with ESG screening reduced excess returns; however, from 2014-2017, the authors found that ESG screening created 3.3\% excess returns annually \cite{Bennani2018HowEI}.
%. These results strongly indicate that this is, on the one hand, driven by low-rated smaller companies (“small sinners”) and clean-coated frms with regard to controversies (“silent saints”) on the other hand.

%Among others, Derwall et al. (2005) and Edmans (2011), who link the doing good while doing well-hypothesis with the managerial myopia theory, conclude that short-term investors are unable (or unwilling) to price the long-term benefts of those activities correctly and therefore undervalue stocks of companies with high levels of engagement in environmental or social aspects, leading to higher returns in the long-run for the respective stocks when compared with other stocks.
%with the fndings of Dorfeitner et al. (2018), who conclude that the benefts of socially responsible activities (measured by the abnormal stock returns) are produced by unexpected additional cash fows which occur mid-to-long term.
%Furthermore, from an investor’s standpoint, having a “clean coat” with regard to controversies is especially proftable for smaller companies, as the absence of these scandals may be overlooked and incorrectly incorporated in the market prices. Thus, one might say that the respective companies “fy under the radar”.