This section will be broke into two components. The first section will adress  theoretical models for asset purchasing and then we will introduce some of the mythology for  EGS factor in the asset pricing framework.\\


 \hl{add a section on the metholoogy for the 2 pass sort use} \cite{Kleibergen2013UnexplainedFA} and \cite{Fama1973RiskRA} \\
The modern foundation of asset pricing began with the capital asset pricing model based on Sharp's 1964's work which used a regression analysis to relate market returns to individual security returns \cite{Sharpe1964CAPITALAP}. The CAPM model was later iterated on by Fama french in 1993 with the 3 factor, which considered several other terms to predict stock returns. Fama and French added size and book to market terms to the CAPM model. They found that these three coefficients created a more explanatory model which demonstrated the relationship that value companies(low book to market) and small companies experience larger excess returns \cite{Fama1992TheCO}. Fama and French again innovated on the three-factor model in 2015 by adding two factors to the regression. In this model, Fama and French consider Investment and profitability while holding the other regression stable factors. The five-factor model analysis found that the addition of the investment and profitability factor made the value factor redundant \cite{Fama2013AFA}. The five-factor framework analysis has been recreated on international exchanges to test the effects on foreign markets. In recreation by Fama French and another study by Zhong and Li, the general relationships demonstrated in the five-factor model are also applicable in China, Japan, and Australian markets \cite{Fama2015InternationalTO, LIN2017141, Chiah2016A}. However, when the Fama French regression was applied to the Australian equity market, the authors did not find the significance of the value factor to subside like was visible in the 2015 paper \cite{Chiah2016A}.  In addition to expanding the five-factor analysis geographically, some scholars have expanded the five-factor model by adding additional terms to assess momentum effects.  In a paper published by  Dirkx and  Peter, the authors expanded the five-factor model by applying the Fama French framework to the German market while also adding a sixth term to assess the effect of risk premium with the CDAX constituents' momentum. Another study by Gregory and Stead used the sixth factor as an ESG factor to analyze if the ESG effect played a significant role in determining risk premia. This paper found that  "the risk premium of the sustainability factor is positive and significant\cite{Gregory2020TheGP}."

In addition to Gregory and Stead's paper, several other authors have used the Fama French model to address the question concerning ESG and risk premia. To set up an ESG Fama French model, many academic journal articles use an index as a positive ESG portfolio and another index as a negative portfolio to create the ESG factor. In Gregory and Stead's paper, the author uses the S\&P ESG index as the high ESG index and the S\&P petroleum index as the low index to create a factor by subtracting the returns from the high ESG portfolio from the low ESG portfolio \cite{Gregory2020TheGP}. In Dorfleitner et al.'s paper, the authors collect Thomas Reuter's controversies and combined ESG score for 2500 companies. With the ranked ESG score data, the author then creates two portfolios by separating the top decile score and the bottom deciles scores, which act as the ESG and contra ESG portfolios. Similar to Gregory et al. paper, the author subtract the top 10\% from the bottom 10\% to create the ESG factor \cite{Dorfleitner2020ESGCA}.   

\hl{dicuss what goes into the MSIC index and why this is an index worth looking at }
%https://www.msci.com/eqb/methodology/meth_docs/MSCI_ESG_Leaders_Methodology_Nov2020.pdf
