As discussed in section \ref{Results}, additional tests are required to ensure the methodology's robustness. The first robustness check I will run is a repetition of the two-pass sort for where the ESG factor is calculated asMSCI ESG index minus the petroleum index (EMP). Table 6 confirms the statistical significance of the market term, the SMB term, and the CMA term. All three of these terms were found to be statistically significant in the initial regression as well. Beyond the coefficients,  the results in table 6 demonstrate a slightly lower $R^2$ for both the Beta decile and the size decile variants, demonstrating that the EMP term contributes to the explanation of excess returns. In addition, the constant term for both size and beta decile portfolios increased in magnitude as a result of omitting the EMP. As the coefficient term gets closer to zero, the factors do a better job describing the returns and indicate that there is no excess return that is not explained. 
 %as dicussed in section \ref{Data} [talk about using mining instead of petrolum ]
 
 \begin{center}
    \paperspacingnarrow
    \begin{tabular}{lcc} \hline
 & (1) & (2) \\
VARIABLES & Size Portfolio & Beta Portfolio \\ \hline
 &  &  \\
Market & 0.993*** & 1.012*** \\
 & (0.0434) & (0.160) \\
HML & 0.0810 & 0.120 \\
 & (0.0628) & (0.0698) \\
SMB & 0.511*** & 0.478*** \\
 & (0.0715) & (0.109) \\
RMW & 0.0509 & 0.0520 \\
 & (0.0312) & (0.0819) \\
CMA & 0.204** & 0.171** \\
 & (0.0639) & (0.0645) \\
Constant & -0.000532 & -0.00133 \\
 & (0.00208) & (0.000942) \\
 &  &  \\
Observations & 1,150 & 1,150 \\
R-squared & 0.855 & 0.846 \\
 Number of groups & 10 & 10 \\ \hline
\multicolumn{3}{c}{ Standard errors in parentheses} \\
\multicolumn{3}{c}{ *** p$<$0.01, ** p$<$0.05, * p$<$0.1} \\
\end{tabular}
\\
    \textbf{Table 6:} Shows  second pass results in a teo pass sort for the size and beta decile portfolios without EMP factor\\
    \paperspacingwide
\end{center}
Another robustness check that is essential to consider is the time-based effects on the model. Although we have demonstrated that the EMP factor is statistically significant and explains part of our model, it is essential to understand if this is ubiquitous across time in the sample or if the  ESG factor has shifted in risk premia over time. To address this robustness concern, I have repeated the two-pass sort with the size decile portfolio by adding a time-based dummy variable. Table 7 details the results from this regression. In 2015 2016, 2018, and 2019 there is statistical significance for the EMP term and a positive association between EMP and excess returns. The ESG factor showed the most strength in 2019 as well as the greatest coefficient value. In 2019 a one-unit change in beta would result in a 25.9\% increase in excess returns. The coefficient value of the EMP term in 2017  holds a negative relationship as a one-unit change in ESG beta resulted in a reduction in excess returns by 10.5\%. Although in 2020, there was a positive relationship, the coefficient is not statistically significant. However, in 2020, the CMA factor, which has been statistically significant at the 1\% level for all other tests, also does not appear to be statistically significant. 2020 represents a bit of a deviation due to the larger market condition from the COVID-19 pandemic rather than the EMP factor's long-run behavior.


\begin{center}
    \paperspacingnarrow
    \begin{tabular}{lcccccc} \hline
 & (1) & (2) & (3) & (4) & (5) & (6) \\
VARIABLES & 2015 & 2016 & 2017 & 2018 & 2019 & 2020 \\ \hline
 &  &  &  &  &  &  \\
Market & 0.881*** & 1.434*** & 0.944*** & 0.932*** & 0.954*** & 1.104*** \\
 & (0.0921) & (0.0584) & (0.141) & (0.0585) & (0.0437) & (0.0493) \\
HML & -0.417*** & -0.377** & 0.0548 & 0.0158 & -0.0806 & 0.132 \\
 & (0.101) & (0.137) & (0.0798) & (0.0776) & (0.0895) & (0.0807) \\
SMB & 0.322** & -0.113 & 0.143 & 0.569*** & 0.483*** & 0.551*** \\
 & (0.104) & (0.100) & (0.131) & (0.115) & (0.109) & (0.163) \\
RMW & 0.400*** & -0.465*** & -0.115 & 0.112 & 0.00982 & -0.267 \\
 & (0.0546) & (0.132) & (0.118) & (0.0964) & (0.120) & (0.172) \\
CMA & 0.810*** & 1.047*** & 0.753** & 0.506*** & 0.559*** & 0.175 \\
 & (0.167) & (0.249) & (0.264) & (0.149) & (0.157) & (0.128) \\
EMP & 0.158*** & 0.0432* & -0.105** & 0.0525* & 0.249*** & 0.00461 \\
 & (0.0397) & (0.0212) & (0.0412) & (0.0256) & (0.0688) & (0.0191) \\
Constant & -0.00170 & 0.00628 & 0.00155 & -0.00372* & -0.00394*** & 0.00947 \\
 & (0.00107) & (0.00428) & (0.00135) & (0.00195) & (0.000811) & (0.0119) \\
 &  &  &  &  &  &  \\
Observations & 120 & 120 & 120 & 120 & 120 & 120 \\
R-squared & 0.916 & 0.896 & 0.864 & 0.975 & 0.935 & 0.950 \\
 Number of groups & 10 & 10 & 10 & 10 & 10 & 10 \\ \hline
\multicolumn{7}{c}{ Standard errors in parentheses} \\
\multicolumn{7}{c}{ *** p$<$0.01, ** p$<$0.05, * p$<$0.1} \\
\end{tabular}
\\
    \textbf{Table 7:} Shows the results for the second pass in a two pass sort. The two pass sort used size based deciles for the first pass. Columns (1)-(6) represent individual years of data from 2016-2020. 
    
    \paperspacingwide
\end{center}

 
 Moving on to the second set of analyses with a slightly different definition of the ESG factor.  As described in the mythology and the results, I created another definition for measuring the effect of ESG by subtracting the Best 10\% of frim from the bottom 10\% of firms as defined by MSCI ESG scores. As demonstrated in the results section, there was no measured statistical relationship between this factor and the excess returns. To further break down this relationship, I decomposed the excess returns on a first digit SIC level to assess if some industries are more sensitive to ESG factors than others. Some industries are likely to be inherently ESG unfriendly and might not have a substantial effect from an ESG factor. To create this analysis, I made a unique ESG factor for each of the SIC first digit industry codes by subtracting returns from the best 10\% of the firm's ESG score from the bottom 10\% of the firm's ESG scores. This process is also detailed in the data section. With the new ESG factor, I regressed a two-pass sort using size decile portfolios on each SIC industry to access abnormal returns. MSCI did not have data on firms that started with SIC codes 0 or  9. Additionally, the MSCI data for SIC code 5 was only BBB; as such, there was no delineation in ESG rating, and I was unable to create a high low ESG factor. Table 8 demonstrates the results from this new set of regressions. For all of the available level SIC portfolios, the market term and the SMB term are highly statistically significant at the 1\% level, which is consistent with the results from both the EMP and the initial MSCI ESG rating analysis. The ESG factor in each of the SIC regressions shows a positive but statistically insignificant coefficient even at the 10\% level. This result suggests that the EGS factor for the MSCI high low ESG factor is priced into the market, unlike the EMP factor from the first part of the analysis. 
\begin{center}
    \paperspacingnarrow
    \begin{tabular}{lccccccc} \hline
 & (1) & (2) & (3) & (4) & (5) & (6) & (7) \\
VARIABLES & SIC 1 & SIC 2 & SIC 3 & SIC 4 & SIC 6 & SIC 7 & SIC 8 \\ \hline
 &  &  &  &  &  &  &  \\
market\_minus\_rf & 1.611*** & 1.166*** & 1.186*** & 1.087*** & 0.736*** & 1.168*** & 1.121*** \\
 & (0.0440) & (0.0422) & (0.0385) & (0.0729) & (0.0650) & (0.0273) & (0.0534) \\
hml & 0.270 & -0.218 & 0.0708 & 0.153** & 0.171*** & 0.0553 & 0.0150 \\
 & (0.161) & (0.174) & (0.0671) & (0.0502) & (0.0422) & (0.0641) & (0.0720) \\
smb & 0.860*** & 0.794*** & 0.773*** & 0.414*** & 0.275*** & 0.753*** & 0.445** \\
 & (0.131) & (0.173) & (0.151) & (0.106) & (0.0464) & (0.155) & (0.185) \\
rmw & -0.0342 & -0.357 & -0.110 & -0.446* & 0.158* & -0.0415 & -0.577** \\
 & (0.122) & (0.325) & (0.111) & (0.225) & (0.0745) & (0.0584) & (0.232) \\
cma & 1.482*** & 0.569*** & 0.374** & 0.549*** & 0.0226 & -0.152 & -0.0517 \\
 & (0.174) & (0.0983) & (0.163) & (0.150) & (0.0743) & (0.157) & (0.210) \\
emp & 0.0999*** & 0.109*** & 0.0567* & 0.0242 & -0.0132 & -0.0842** & -0.0511 \\
 & (0.0239) & (0.0335) & (0.0253) & (0.0512) & (0.0114) & (0.0364) & (0.0381) \\
Constant & -0.0103*** & 0.00164 & 0.00404* & 0.000490 & -0.000708 & 0.00707* & -0.00268 \\
 & (0.00204) & (0.00202) & (0.00183) & (0.00375) & (0.00139) & (0.00328) & (0.00201) \\
 &  &  &  &  &  &  &  \\
Observations & 1,024 & 1,018 & 1,023 & 1,020 & 1,045 & 1,016 & 1,001 \\
R-squared & 0.596 & 0.689 & 0.745 & 0.575 & 0.767 & 0.699 & 0.559 \\
 Number of groups & 10 & 10 & 10 & 10 & 10 & 10 & 10 \\ \hline
\multicolumn{8}{c}{ Standard errors in parentheses} \\
\multicolumn{8}{c}{ *** p$<$0.01, ** p$<$0.05, * p$<$0.1} \\
\end{tabular}
\\
    \textbf{Table 8:} shows the reuslts for the second regression reuslts in a two pass sort. Columns (1)-(7) represent the first digit SIC code industries and their respective sensitivity to the fama french five factor as well as the ESG factor. SIC code 1 covers the mining and construction industry, SIC codes 2-3 covers manufacturing industry, SIC code 4 covers Transportation, Communications, Electric, Gas and Sanitary service SIC code 6 covers Finance, Insurance, And Real Estate SIC code 7-8 covers services\\
    
    \paperspacingwide
\end{center}
 
A final robustness check is conducted to address another concern about the MSCI ESG data. As explained in the dad section, the ESG data from MSCI to create the ESG high low index is static. That is to say, the data for ESG scores is only a snapshot of ESG ratings for April 2021. The use of the entire window from June 2011 to March 2021 to assess excess returns might be a bad characterization of the ESG landscape as a company that could have been a low ESG company might have switched to a high ESG company time. To address this concern, I recreated the regression from table 5 by limiting the time horizon to 2018-2020. Although there could have been some movement in the ESG scores during the modified time frame, a companies ability to respond to change has a degree of stickiness as corporate initiatives often take several years to develop, and therefore some of the ESG factors would have a lag effect.  Looking at the results from table 9, the Market SMB and CMA are statistically significant, which was also observed in table 5. The ESG factor in table 9, similar to table 5, is not statistically significant for either the beta or size decile portfolios. The magnitude of the  ESG factor for the size-based portfolio has increased from .004 to .013; however, the beta decile portfolio has changed signs from .0145 to -.013. The muddled resulted mixed with the statistical insignificance do not provide evidence of a shorter time horizon improving the accuracy of the ESG factor as a predictor of stock performance. An additional source of error could be a result of the sample bias associated with the companies MSCI has rated. MSIC has only rated 545 of the US public equities. Of the rated companies 439 (or99\%) have a market cap greater than 1B.As a result of the large market cap bias, the ESG factor does not accurately describe all of the ESG risk associated with the market. 
 
 \begin{center}
    \paperspacingnarrow
    \begin{tabular}{lcc} \hline
 & (1) & (2) \\
VARIABLES & Size portfolio & Beta Portfolio \\ \hline
 &  &  \\
market\_minus\_rf & 1.026*** & 1.039*** \\
 & (0.0380) & (0.145) \\
hml & 0.0289 & 0.0344 \\
 & (0.0576) & (0.0361) \\
smb & 0.654*** & 0.676*** \\
 & (0.0917) & (0.165) \\
rmw & -0.0379 & 0.000565 \\
 & (0.110) & (0.0549) \\
cma & 0.341*** & 0.387** \\
 & (0.0704) & (0.120) \\
emp & -0.0233 & 0.0419** \\
 & (0.0170) & (0.0162) \\
Constant & 0.000308 & -0.00126 \\
 & (0.00272) & (0.00141) \\
 &  &  \\
Observations & 360 & 360 \\
R-squared & 0.922 & 0.921 \\
 Number of groups & 10 & 10 \\ \hline
\multicolumn{3}{c}{ Standard errors in parentheses} \\
\multicolumn{3}{c}{ *** p$<$0.01, ** p$<$0.05, * p$<$0.1} \\
\end{tabular}
\\
    \textbf{Table 9:} Shows the same regression from table 5 with an altered time frame to only include data from 2018-2020
    
    \paperspacingwide
\end{center}
 
 

