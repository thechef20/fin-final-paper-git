

when doing the SIC porlfios did not have data for SIC starting with 5 b/c they were all BB companies 
 %as dicussed in section \ref{Data} [talk about using mining instead of petrolum ]
 
 \begin{center}
    \paperspacingnarrow
    \begin{tabular}{lcc} \hline
 & (1) & (2) \\
VARIABLES & Size Portfolio & Beta Portfolio \\ \hline
 &  &  \\
Market & 0.993*** & 1.012*** \\
 & (0.0434) & (0.160) \\
HML & 0.0810 & 0.120 \\
 & (0.0628) & (0.0698) \\
SMB & 0.511*** & 0.478*** \\
 & (0.0715) & (0.109) \\
RMW & 0.0509 & 0.0520 \\
 & (0.0312) & (0.0819) \\
CMA & 0.204** & 0.171** \\
 & (0.0639) & (0.0645) \\
Constant & -0.000532 & -0.00133 \\
 & (0.00208) & (0.000942) \\
 &  &  \\
Observations & 1,150 & 1,150 \\
R-squared & 0.855 & 0.846 \\
 Number of groups & 10 & 10 \\ \hline
\multicolumn{3}{c}{ Standard errors in parentheses} \\
\multicolumn{3}{c}{ *** p$<$0.01, ** p$<$0.05, * p$<$0.1} \\
\end{tabular}
\\
    \textbf{Table 6:} second pass results for the size and beta decile portfolios without ESG factor\\
    \paperspacingwide
\end{center}

 
 dicuss limitation of only one set of ESG factors for MSCI 
 
 we used the Size deciles on a go froward basis when analysis the data as the statistical relationship was more apparent 
 
 \textbf{Analysis for the w/o ESG on the data}\\
 \textbf{yearly effect}\\
\begin{center}
    \paperspacingnarrow
    \begin{tabular}{lcccccc} \hline
 & (1) & (2) & (3) & (4) & (5) & (6) \\
VARIABLES & 2015 & 2016 & 2017 & 2018 & 2019 & 2020 \\ \hline
 &  &  &  &  &  &  \\
Market & 0.881*** & 1.434*** & 0.944*** & 0.932*** & 0.954*** & 1.104*** \\
 & (0.0921) & (0.0584) & (0.141) & (0.0585) & (0.0437) & (0.0493) \\
HML & -0.417*** & -0.377** & 0.0548 & 0.0158 & -0.0806 & 0.132 \\
 & (0.101) & (0.137) & (0.0798) & (0.0776) & (0.0895) & (0.0807) \\
SMB & 0.322** & -0.113 & 0.143 & 0.569*** & 0.483*** & 0.551*** \\
 & (0.104) & (0.100) & (0.131) & (0.115) & (0.109) & (0.163) \\
RMW & 0.400*** & -0.465*** & -0.115 & 0.112 & 0.00982 & -0.267 \\
 & (0.0546) & (0.132) & (0.118) & (0.0964) & (0.120) & (0.172) \\
CMA & 0.810*** & 1.047*** & 0.753** & 0.506*** & 0.559*** & 0.175 \\
 & (0.167) & (0.249) & (0.264) & (0.149) & (0.157) & (0.128) \\
EMP & 0.158*** & 0.0432* & -0.105** & 0.0525* & 0.249*** & 0.00461 \\
 & (0.0397) & (0.0212) & (0.0412) & (0.0256) & (0.0688) & (0.0191) \\
Constant & -0.00170 & 0.00628 & 0.00155 & -0.00372* & -0.00394*** & 0.00947 \\
 & (0.00107) & (0.00428) & (0.00135) & (0.00195) & (0.000811) & (0.0119) \\
 &  &  &  &  &  &  \\
Observations & 120 & 120 & 120 & 120 & 120 & 120 \\
R-squared & 0.916 & 0.896 & 0.864 & 0.975 & 0.935 & 0.950 \\
 Number of groups & 10 & 10 & 10 & 10 & 10 & 10 \\ \hline
\multicolumn{7}{c}{ Standard errors in parentheses} \\
\multicolumn{7}{c}{ *** p$<$0.01, ** p$<$0.05, * p$<$0.1} \\
\end{tabular}
\\
    \textbf{Table 7:} Time based cofficents
    \paperspacingwide
\end{center}
 \textbf{MSCI data 2018-2020}\\
 \begin{center}
    \paperspacingnarrow
\begin{tabular}{lcc} \hline
 & (1) & (2) \\
VARIABLES & Size portfolio & Beta Portfolio \\ \hline
 &  &  \\
market\_minus\_rf & 1.026*** & 1.039*** \\
 & (0.0380) & (0.145) \\
hml & 0.0289 & 0.0344 \\
 & (0.0576) & (0.0361) \\
smb & 0.654*** & 0.676*** \\
 & (0.0917) & (0.165) \\
rmw & -0.0379 & 0.000565 \\
 & (0.110) & (0.0549) \\
cma & 0.341*** & 0.387** \\
 & (0.0704) & (0.120) \\
emp & -0.0233 & 0.0419** \\
 & (0.0170) & (0.0162) \\
Constant & 0.000308 & -0.00126 \\
 & (0.00272) & (0.00141) \\
 &  &  \\
Observations & 360 & 360 \\
R-squared & 0.922 & 0.921 \\
 Number of groups & 10 & 10 \\ \hline
\multicolumn{3}{c}{ Standard errors in parentheses} \\
\multicolumn{3}{c}{ *** p$<$0.01, ** p$<$0.05, * p$<$0.1} \\
\end{tabular}
\\
\textbf{Table XX:} SIC top 10\% minus bottom 10\% 2018-2020
    \paperspacingwide
\end{center}
 \textbf{MSCI SIC level analysis}\\
 
  \begin{center}
    \paperspacingnarrow
    \begin{tabular}{lccccccc} \hline
 & (1) & (2) & (3) & (4) & (5) & (6) & (7) \\
VARIABLES & SIC 1 & SIC 2 & SIC 3 & SIC 4 & SIC 6 & SIC 7 & SIC 8 \\ \hline
 &  &  &  &  &  &  &  \\
market\_minus\_rf & 1.611*** & 1.166*** & 1.186*** & 1.087*** & 0.736*** & 1.168*** & 1.121*** \\
 & (0.0440) & (0.0422) & (0.0385) & (0.0729) & (0.0650) & (0.0273) & (0.0534) \\
hml & 0.270 & -0.218 & 0.0708 & 0.153** & 0.171*** & 0.0553 & 0.0150 \\
 & (0.161) & (0.174) & (0.0671) & (0.0502) & (0.0422) & (0.0641) & (0.0720) \\
smb & 0.860*** & 0.794*** & 0.773*** & 0.414*** & 0.275*** & 0.753*** & 0.445** \\
 & (0.131) & (0.173) & (0.151) & (0.106) & (0.0464) & (0.155) & (0.185) \\
rmw & -0.0342 & -0.357 & -0.110 & -0.446* & 0.158* & -0.0415 & -0.577** \\
 & (0.122) & (0.325) & (0.111) & (0.225) & (0.0745) & (0.0584) & (0.232) \\
cma & 1.482*** & 0.569*** & 0.374** & 0.549*** & 0.0226 & -0.152 & -0.0517 \\
 & (0.174) & (0.0983) & (0.163) & (0.150) & (0.0743) & (0.157) & (0.210) \\
emp & 0.0999*** & 0.109*** & 0.0567* & 0.0242 & -0.0132 & -0.0842** & -0.0511 \\
 & (0.0239) & (0.0335) & (0.0253) & (0.0512) & (0.0114) & (0.0364) & (0.0381) \\
Constant & -0.0103*** & 0.00164 & 0.00404* & 0.000490 & -0.000708 & 0.00707* & -0.00268 \\
 & (0.00204) & (0.00202) & (0.00183) & (0.00375) & (0.00139) & (0.00328) & (0.00201) \\
 &  &  &  &  &  &  &  \\
Observations & 1,024 & 1,018 & 1,023 & 1,020 & 1,045 & 1,016 & 1,001 \\
R-squared & 0.596 & 0.689 & 0.745 & 0.575 & 0.767 & 0.699 & 0.559 \\
 Number of groups & 10 & 10 & 10 & 10 & 10 & 10 & 10 \\ \hline
\multicolumn{8}{c}{ Standard errors in parentheses} \\
\multicolumn{8}{c}{ *** p$<$0.01, ** p$<$0.05, * p$<$0.1} \\
\end{tabular}
\\
    \textbf{Table 6:} second pass results for the size and beta decile portfolios without ESG factor\\
    \paperspacingwide
\end{center}

 WE ADJUSTED THE YEAR to 2019!!
 problems we can never solve for both of these indexes does not do a great job capturing small companies so size might be a bit of a probelm overall as the small companies do not have ther ESG scores baked in as MSCI does not have the resources to rank every company