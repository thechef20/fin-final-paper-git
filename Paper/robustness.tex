As discussed in section \ref{Results}, additional tests are required to ensure the robustness of the methodology. The first robustness check I will run is a repetition of the two-pass sort for where the ESG factor is calculated as MSCI ESG index minus the petroleum index (EMP). Table 6 confirms the statistical significance of the market term, the SMB term, and the CMA term. Beyond the coefficients,  the results in table 6 demonstrate a slightly lower $R^2$ for both the Beta decile and the size decile variants, demonstrating that the EMP term contributes to the explanation of excess returns. In addition, the constant term for both size and beta decile portfolios increased in magnitude as a result of omitting the EMP. As the coefficient term gets closer to zero, the factors do a better job describing the excess return and relationship and eliminates all other sources of unexplained excess returns.  
 
 \begin{center}
    \paperspacingnarrow
    \begin{tabular}{lcc} \hline
 & (1) & (2) \\
VARIABLES & Size Portfolio & Beta Portfolio \\ \hline
 &  &  \\
Market & 0.993*** & 1.012*** \\
 & (0.0434) & (0.160) \\
HML & 0.0810 & 0.120 \\
 & (0.0628) & (0.0698) \\
SMB & 0.511*** & 0.478*** \\
 & (0.0715) & (0.109) \\
RMW & 0.0509 & 0.0520 \\
 & (0.0312) & (0.0819) \\
CMA & 0.204** & 0.171** \\
 & (0.0639) & (0.0645) \\
Constant & -0.000532 & -0.00133 \\
 & (0.00208) & (0.000942) \\
 &  &  \\
Observations & 1,150 & 1,150 \\
R-squared & 0.855 & 0.846 \\
 Number of groups & 10 & 10 \\ \hline
\multicolumn{3}{c}{ Standard errors in parentheses} \\
\multicolumn{3}{c}{ *** p$<$0.01, ** p$<$0.05, * p$<$0.1} \\
\end{tabular}
\\
    \textbf{Table 6:} shows  second pass results in a teo pass sort for the size and beta decile portfolios without EMP factor\\
    \paperspacingwide
\end{center}
Another robustness check that is essential to consider is the time-based effects on the model. Although I have demonstrated that the EMP factor is statistically significant and explains part of our model, it is essential to understand if this is ubiquitous across time in the sample or if the  ESG factor has shifted in risk premia over time. To address this robustness concern, I have repeated the two-pass sort with the size decile portfolio by adding a time-based dummy variable. Table 7 details the results from this regression. In 2015 2016, 2018, and 2019 there is statistical significance for the EMP term and a positive association between EMP and excess returns. The ESG factor showed the most statistical strength in 2019 with the greatest excess return. In 2019 a one-unit change in beta would result in a 25.9\% increase in excess returns. These results are consistent with Bennani et al.'s paper decomposing risk premia over time for 2015 and 2016. The coefficient value of the EMP term in 2017  holds a negative relationship as a one-unit change in ESG beta resulted in a reduction in excess returns by 10.5\%. In 2020, there was a positive relationship, however the coefficient is not statistically significant. The CMA factor, which has been statistically significant at the 1\% level for all other tests, also does not appear to be statistically significant for 2020. The 2020 calendar year represents a bit of a deviation from typical trends likely due to the larger market condition from the COVID-19 pandemic rather than the EMP factor's long-run behavior. In addition to this set of robustness checks, it would be beneficial to repeat the series of tests replacing the S\&P petroleum index with the S\&P mining index as the contra ESG portfolio. Both of the proposed indexes fall under the first digit  SIC code 1  and are exposed to similar ESG risks; however, the additional test can isolate risk premia and test if the factor is capturing petroleum industry effect or a robust ESG factor.  


\begin{center}
    \paperspacingnarrow
    \begin{tabular}{lcccccc} \hline
 & (1) & (2) & (3) & (4) & (5) & (6) \\
VARIABLES & 2015 & 2016 & 2017 & 2018 & 2019 & 2020 \\ \hline
 &  &  &  &  &  &  \\
Market & 0.881*** & 1.434*** & 0.944*** & 0.932*** & 0.954*** & 1.104*** \\
 & (0.0921) & (0.0584) & (0.141) & (0.0585) & (0.0437) & (0.0493) \\
HML & -0.417*** & -0.377** & 0.0548 & 0.0158 & -0.0806 & 0.132 \\
 & (0.101) & (0.137) & (0.0798) & (0.0776) & (0.0895) & (0.0807) \\
SMB & 0.322** & -0.113 & 0.143 & 0.569*** & 0.483*** & 0.551*** \\
 & (0.104) & (0.100) & (0.131) & (0.115) & (0.109) & (0.163) \\
RMW & 0.400*** & -0.465*** & -0.115 & 0.112 & 0.00982 & -0.267 \\
 & (0.0546) & (0.132) & (0.118) & (0.0964) & (0.120) & (0.172) \\
CMA & 0.810*** & 1.047*** & 0.753** & 0.506*** & 0.559*** & 0.175 \\
 & (0.167) & (0.249) & (0.264) & (0.149) & (0.157) & (0.128) \\
EMP & 0.158*** & 0.0432* & -0.105** & 0.0525* & 0.249*** & 0.00461 \\
 & (0.0397) & (0.0212) & (0.0412) & (0.0256) & (0.0688) & (0.0191) \\
Constant & -0.00170 & 0.00628 & 0.00155 & -0.00372* & -0.00394*** & 0.00947 \\
 & (0.00107) & (0.00428) & (0.00135) & (0.00195) & (0.000811) & (0.0119) \\
 &  &  &  &  &  &  \\
Observations & 120 & 120 & 120 & 120 & 120 & 120 \\
R-squared & 0.916 & 0.896 & 0.864 & 0.975 & 0.935 & 0.950 \\
 Number of groups & 10 & 10 & 10 & 10 & 10 & 10 \\ \hline
\multicolumn{7}{c}{ Standard errors in parentheses} \\
\multicolumn{7}{c}{ *** p$<$0.01, ** p$<$0.05, * p$<$0.1} \\
\end{tabular}
\\
    \textbf{Table 7:} shows the results for the second pass in a two pass sort. The two pass sort used size based deciles for the first pass. Columns (1)-(6) represent individual years of data from 2015-2020. 
    \paperspacingwide
\end{center}

 
 Moving on to the second set of robustness checks with a slightly different definition of the ESG factor.  As described in the methodology and the results, I created another definition for measuring the effect of ESG by subtracting the Best 10\% of frim from the bottom 10\% of firms as defined by MSCI ESG scores. As demonstrated in the results section, there was no measured statistical relationship between this factor and the excess returns. To further break down this relationship, I decomposed the excess returns on a first digit SIC level to assess if some industries are more sensitive to ESG factors than others. Some industries are likely to be inherently ESG unfriendly and might not be substantially affected by an ESG factor. 

To create this analysis, I repeated the two-pass sort methodology from the results using the size decile portfolios and isolated the returns for each of the first digit SIC code industries.  I regressed the same ESG factor described in the results on this subset to assess the different effects the factor has on each industry. As the MSCI ESG rating dataset for this analysis did not have robust data for industries with first digit SIC codes 0, 5, or 9, I elected not to analyze these SIC industries as the construction of the factor did not account for them. 

Table 8 demonstrates the results from this new set of regressions. For all of the available  SIC portfolios, the market term and the SMB term are highly statistically significant at the 1\% level, which is consistent with the results from both the EMP and the initial MSCI ESG rating analysis. The ESG factor response to each industry is not uniform. Industries with first digit SIC codes 3,4,6,7, and 8 have small non statistically significant coefficients.  The first digit SIC code 2 demonstrated a statistically significant negative relationship to the ESG factor, demonstrating that the high ESG beta portfolio would harm excess returns. However, the first digit SIC code 1 demonstrated a highly statistically significant positive coefficient.  A one-unit increase in ESG beta would result in an 11.6\% increase in excess returns for first digit SIC industry 1. These results confirm results from part 1 of the analyses with the EMP factor as the SIC code 1 encompasses the petroleum industry. Future tests should be constructed to ensure that the ESG factor is picking up on more than negative excess returns from investing in the petroleum industry. 

\begin{center}
    \paperspacingnarrow
    \begin{tabular}{lccccccc} \hline
 & (1) & (2) & (3) & (4) & (5) & (6) & (7) \\
VARIABLES & SIC 1 & SIC 2 & SIC 3 & SIC 4 & SIC 6 & SIC 7 & SIC 8 \\ \hline
 &  &  &  &  &  &  &  \\
Market & 1.586*** & 1.130*** & 1.166*** & 1.063*** & 0.743*** & 1.158*** & 1.060*** \\
 & (0.0505) & (0.0462) & (0.0443) & (0.0632) & (0.0638) & (0.0348) & (0.0721) \\
HML & 0.351* & -0.160 & 0.120 & 0.176*** & 0.140** & 0.00324 & 0.0610 \\
 & (0.157) & (0.148) & (0.0780) & (0.0414) & (0.0465) & (0.0677) & (0.0706) \\
SMB & 0.816*** & 0.740*** & 0.760*** & 0.409*** & 0.255*** & 0.772*** & 0.494*** \\
 & (0.111) & (0.175) & (0.145) & (0.108) & (0.0451) & (0.159) & (0.144) \\
RMW & 0.000434 & -0.306 & -0.115 & -0.373 & 0.147* & -0.0377 & -0.522** \\
 & (0.0910) & (0.287) & (0.103) & (0.213) & (0.0707) & (0.0577) & (0.188) \\
CMA & 0.804*** & 0.325*** & 0.212 & 0.350* & 0.0153 & -0.115 & -0.0508 \\
 & (0.148) & (0.0753) & (0.134) & (0.163) & (0.0410) & (0.141) & (0.179) \\
ESG & 0.116*** & -0.0748* & 0.0100 & -0.0176 & -0.0160 & -0.00269 & 0.0321 \\
 & (0.0262) & (0.0353) & (0.0106) & (0.0447) & (0.0111) & (0.0397) & (0.0324) \\
Constant & -0.0133*** & 0.00155 & 0.00314 & 0.000687 & -0.00106 & 0.00757* & -0.00106 \\
 & (0.00251) & (0.00267) & (0.00195) & (0.00389) & (0.00133) & (0.00336) & (0.00128) \\
 &  &  &  &  &  &  &  \\
Observations & 1,150 & 1,150 & 1,150 & 1,147 & 1,150 & 1,150 & 1,138 \\
R-squared & 0.583 & 0.686 & 0.745 & 0.574 & 0.762 & 0.693 & 0.549 \\
 Number of groups & 10 & 10 & 10 & 10 & 10 & 10 & 10 \\ \hline
\multicolumn{8}{c}{ Standard errors in parentheses} \\
\multicolumn{8}{c}{ *** p$<$0.01, ** p$<$0.05, * p$<$0.1} \\
\end{tabular}
\\
    \textbf{Table 8:} shows the results for the second regression results in a two-pass sort. Columns (1)-(7) represent the first digit SIC code industries and their respective risk premia to the Fama French five-factor as well as the ESG factor. SIC code 1 covers the mining and construction industry, SIC codes 2-3 covers manufacturing industry, SIC code 4 covers transportation, communications, electric, gas, and sanitary service SIC code 6 covers finance, insurance, and real estate SIC code 7-8 covers services\\
    
    \paperspacingwide
\end{center}
 
A final robustness check is conducted to address another concern about the MSCI ESG data. As explained in the data section, the MSCI ESG data used to create the factor is static. That is to say, the data for ESG scores is only a snapshot of ESG ratings for April 2021. The use of the entire window from June 2011 to March 2021 to assess excess returns might be a bad characterization of the ESG landscape as a company that could have been a low ESG company might have switched to a high ESG company in that time. To address this concern, I recreated the regression from table 5 by limiting the time horizon to 2018-2021. Although there could have been some movement in the ESG scores during the modified time frame, a companies ability to respond to ESG objectives has a degree of stickiness as corporate initiatives often take several years to develop and implement, leading to a lag effect.  Looking at the results from table 9, the market, SMB, and CMA factors are all statistically significant, which was also observed in table 5. The ESG factor in table 9, similar to table 5, is not statistically significant for either the beta or size decile portfolios. The magnitude of the  ESG factor for the size-based portfolio has increased from .004 to .013, which indicates the EGS factor might have a larger impact on excess returns. However, the beta decile portfolio has changed signs from .0145 to -.013, which indicates that the EGS factor has a negative impact on excess returns. The muddled resulted mixed with the statistical insignificance do not provide evidence of a shorter time horizon improving the accuracy of the ESG factor as a predictor of stock performance. An additional source of error could be a result of the sample bias associated with the companies MSCI has rated. MSCI's publicly available ESG ratings data has 545 US firms. Of the rated companies, 439 (or 99\% of rated firms) have a market cap greater than \$1 billion. As a result of the large market cap bias, the ESG factor might not accurately reflect the risk premia associated with all ESG investing. 
 
 \begin{center}
    \paperspacingnarrow
    \begin{tabular}{lcc} \hline
 & (1) & (2) \\
VARIABLES & Size portfolio & Beta Portfolio \\ \hline
 &  &  \\
market\_minus\_rf & 1.026*** & 1.039*** \\
 & (0.0380) & (0.145) \\
hml & 0.0289 & 0.0344 \\
 & (0.0576) & (0.0361) \\
smb & 0.654*** & 0.676*** \\
 & (0.0917) & (0.165) \\
rmw & -0.0379 & 0.000565 \\
 & (0.110) & (0.0549) \\
cma & 0.341*** & 0.387** \\
 & (0.0704) & (0.120) \\
emp & -0.0233 & 0.0419** \\
 & (0.0170) & (0.0162) \\
Constant & 0.000308 & -0.00126 \\
 & (0.00272) & (0.00141) \\
 &  &  \\
Observations & 360 & 360 \\
R-squared & 0.922 & 0.921 \\
 Number of groups & 10 & 10 \\ \hline
\multicolumn{3}{c}{ Standard errors in parentheses} \\
\multicolumn{3}{c}{ *** p$<$0.01, ** p$<$0.05, * p$<$0.1} \\
\end{tabular}
\\
    \textbf{Table 9:} shows the same regression from table 5 with an altered time frame to only include data from 2018-2021
    \paperspacingwide
\end{center}
 
 

