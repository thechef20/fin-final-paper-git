\documentclass[12pt,oneside,reqno]{amsart}
\usepackage{mathtools, stackengine}
\numberwithin{equation}{section}
\usepackage{esint}
%\usepackage{undertilde}
\newcommand\norm[1]{\left\lVert#1\right\rVert}
\usepackage{tikz}
\usetikzlibrary{arrows} 
\usetikzlibrary{calc,angles,quotes}
\newtheorem{theorem}{Theorem}[section]
\newtheorem{remark}[theorem]{Remark}
\newtheorem{lemma}[theorem]{Lemma}
\newtheorem{corollary}[theorem]{Corollary}
\usepackage{caption}
\usepackage{subcaption}
\usepackage{longtable}
\usepackage{color,soul}
\usepackage{bm}
\usepackage{setspace}
\newcommand{\ba}{\backslash}
\doublespacing
\usepackage{afterpage}
\usepackage{listings}
\usepackage{geometry} %may be sus
\usepackage[framed,numbered]{matlab-prettifier}
% turn "backspace" character (ASCII 8) into a tab (9) to avoid Matlab warning backspace
\catcode8=9
\lstset{
  style             = Matlab-editor,
  basicstyle        = {\ttfamily\tiny},
  mlshowsectionrules = true,
  upquote           = true
}

\newcommand{\spacer}{\vspace{6mm} \noindent}
\usepackage{parskip}% http://ctan.org/pkg/parskip

\usepackage{enumitem}

% BIB stuff

%\usepackage{biblatex}
%\addbibresource{Matt_Bib.bib}

\usepackage{hyperref}



\definecolor{codeblue}{rgb}{0.29296875, 0.51953125, 0.68359375}
\definecolor{codegreen}{rgb}{0.47265625, 0.62890625, 0.40234375}
\definecolor{codegray}{rgb}{0.95703125, 0.95703125, 0.95703125}
\definecolor{codecrimson}{rgb}{0.87109375,0.3984375,0.3984375}



\lstdefinelanguage{Stata}{
    % System commands
    morekeywords=[1]{regress, reg, summarize, sum, display, di, generate, gen, bysort, use, import, delimited, predict, quietly, probit, margins, test},
    % Reserved words
    morekeywords=[2]{aggregate, array, boolean, break, byte, case, catch, class, colvector, complex, const, continue, default, delegate, delete, do, double, else, eltypedef, end, enum, explicit, export, external, float, for, friend, function, global, goto, if, inline, int, local, long, mata, matrix, namespace, new, numeric, NULL, operator, orgtypedef, pointer, polymorphic, pragma, private, protected, public, quad, real, return, rowvector, scalar, short, signed, static, strL, string, struct, super, switch, template, this, throw, transmorphic, try, typedef, typename, union, unsigned, using, vector, version, virtual, void, volatile, while,},
    % Keywords
    morekeywords=[3]{forvalues, foreach, set},
    % Date and time functions
    morekeywords=[4]{bofd, Cdhms, Chms, Clock, clock, Cmdyhms, Cofc, cofC, Cofd, cofd, daily, date, day, dhms, dofb, dofC, dofc, dofh, dofm, dofq, dofw, dofy, dow, doy, halfyear, halfyearly, hh, hhC, hms, hofd, hours, mdy, mdyhms, minutes, mm, mmC, mofd, month, monthly, msofhours, msofminutes, msofseconds, qofd, quarter, quarterly, seconds, ss, ssC, tC, tc, td, th, tm, tq, tw, week, weekly, wofd, year, yearly, yh, ym, yofd, yq, yw,},
    % Mathematical functions
    morekeywords=[5]{abs, ceil, cloglog, comb, digamma, exp, expm1, floor, int, invcloglog, invlogit, ln, ln1m, ln, ln1p, ln, lnfactorial, lngamma, log, log10, log1m, log1p, logit, max, min, mod, reldif, round, sign, sqrt, sum, trigamma, trunc,},
    % Matrix functions
    morekeywords=[6]{cholesky, coleqnumb, colnfreeparms, colnumb, colsof, corr, det, diag, diag0cnt, el, get, hadamard, I, inv, invsym, issymmetric, J, matmissing, matuniform, mreldif, nullmat, roweqnumb, rownfreeparms, rownumb, rowsof, sweep, trace, vec, vecdiag, },
    % Programming functions
    morekeywords=[7]{autocode, byteorder, c, _caller, chop, abs, clip, cond, e, fileexists, fileread, filereaderror, filewrite, float, fmtwidth, has_eprop, inlist, inrange, irecode, matrix, maxbyte, maxdouble, maxfloat, maxint, maxlong, mi, minbyte, mindouble, minfloat, minint, minlong, missing, r, recode, replay, return, s, scalar, smallestdouble,},
    % Random-number functions
    morekeywords=[8]{rbeta, rbinomial, rcauchy, rchi2, rexponential, rgamma, rhypergeometric, rigaussian, rlaplace, rlogistic, rnbinomial, rnormal, rpoisson, rt, runiform, runiformint, rweibull, rweibullph,},
    % Selecting time-span functions
    morekeywords=[9]{tin, twithin,},
    % Statistical functions
    morekeywords=[10]{betaden, binomial, binomialp, binomialtail, binormal, cauchy, cauchyden, cauchytail, chi2, chi2den, chi2tail, dgammapda, dgammapdada, dgammapdadx, dgammapdx, dgammapdxdx, dunnettprob, exponential, exponentialden, exponentialtail, F, Fden, Ftail, gammaden, gammap, gammaptail, hypergeometric, hypergeometricp, ibeta, ibetatail, igaussian, igaussianden, igaussiantail, invbinomial, invbinomialtail, invcauchy, invcauchytail, invchi2, invchi2tail, invdunnettprob, invexponential, invexponentialtail, invF, invFtail, invgammap, invgammaptail, invibeta, invibetatail, invigaussian, invigaussiantail, invlaplace, invlaplacetail, invlogistic, invlogistictail, invnbinomial, invnbinomialtail, invnchi2, invnF, invnFtail, invnibeta, invnormal, invnt, invnttail, invpoisson, invpoissontail, invt, invttail, invtukeyprob, invweibull, invweibullph, invweibullphtail, invweibulltail, laplace, laplaceden, laplacetail, lncauchyden, lnigammaden, lnigaussianden, lniwishartden, lnlaplaceden, lnmvnormalden, lnnormal, lnnormalden, lnwishartden, logistic, logisticden, logistictail, nbetaden, nbinomial, nbinomialp, nbinomialtail, nchi2, nchi2den, nchi2tail, nF, nFden, nFtail, nibeta, normal, normalden, npnchi2, npnF, npnt, nt, ntden, nttail, poisson, poissonp, poissontail, t, tden, ttail, tukeyprob, weibull, weibullden, weibullph, weibullphden, weibullphtail, weibulltail,},
    % String functions 
    morekeywords=[11]{abbrev, char, collatorlocale, collatorversion, indexnot, plural, plural, real, regexm, regexr, regexs, soundex, soundex_nara, strcat, strdup, string, strofreal, string, strofreal, stritrim, strlen, strlower, strltrim, strmatch, strofreal, strofreal, strpos, strproper, strreverse, strrpos, strrtrim, strtoname, strtrim, strupper, subinstr, subinword, substr, tobytes, uchar, udstrlen, udsubstr, uisdigit, uisletter, ustrcompare, ustrcompareex, ustrfix, ustrfrom, ustrinvalidcnt, ustrleft, ustrlen, ustrlower, ustrltrim, ustrnormalize, ustrpos, ustrregexm, ustrregexra, ustrregexrf, ustrregexs, ustrreverse, ustrright, ustrrpos, ustrrtrim, ustrsortkey, ustrsortkeyex, ustrtitle, ustrto, ustrtohex, ustrtoname, ustrtrim, ustrunescape, ustrupper, ustrword, ustrwordcount, usubinstr, usubstr, word, wordbreaklocale, worcount,},
    % Trig functions
    morekeywords=[12]{acos, acosh, asin, asinh, atan, atanh, cos, cosh, sin, sinh, tan, tanh,},
    morecomment=[l]{//},
    % morecomment=[l]{*},  // `*` maybe used as multiply operator. So use `//` as line comment.
    morecomment=[s]{/*}{*/},
    % The following is used by macros, like `lags'.
    morestring=[b]{`}{'},
    % morestring=[d]{'},
    morestring=[b]",
    morestring=[d]",
    % morestring=[d]{\\`},
    % morestring=[b]{'},
    sensitive=true,
}


\begin{document}
\section{Introduction and Motivation}
Asset pricing has a long history of models and theories governing how investors make decisions. Over the last decade, investors have become increasingly interested in factors outside of traditional financial metrics when assessing the possibility of an investment's returns. Much of this enthusiasm has been based on an ESG, Environmental Social Governance, framework. ESG investing considers how firms respond to these factors as potential impetuses for excess performance. This increased attention is especially visible in Morgan Stanley's 2019 survey of the ESG landscape. Morgan Standley  found that 75\% of assessment managers say that "their firm has adopted sustainable investing." Only two years prior, in 2016, just 10\% of surveyed asset managed had adopted sustainable investing, demonstrating the rapid speed of adoption for ESG investing analysis \cite{morgan_stanley_sustainable_2019}. Practitioners have utilized the ESG framework to reduce risk in their investments while also finding abnormal returns.

Global climate change presents a host of possible and sizable financial risks from increased water levels, which could threaten major global cities' safety, including Miami, Honk Konk, and Shanghai, with the potential to displace 100's of millions of people \cite{holder_three-degree_nodate}. Global climate change is also shifting the landscape for global agricultural production as more extreme weather conditions, including droughts, pose problems to the most fundamental building block of economic activity. In addition to risk mitigation, investors look to utilize ESG principle for superior returns. A research report published by McKinsey and Company linked gender diversity on executive teams to higher profitability and superior value creation \cite{mckinsey_diversity}.

In addition to risk investing professionals, ESG investing is appealing to a younger set of investors interested in a plethora of global issues from global climate change to diversity and inclusion in the workplace. A study conducted by Morgan Stanly found that millennials investors were twice as likely as the general investing population to invest in a company with social or environmental goals.

This new wave of investor sentiment has created strong tailwinds for sustainable investment projects. In 2020 alone, fund flows into sustainable funds grew by 51 billion dollars which is up over 100\% for the 2019 fund flow into sustainable investment products of about 21, placing total assets under management in sustainable products at over 250B. Fund flows into sustainable products for 2020 accounted for a little less a quarter of all US mutual fund flow and about one-fifteenth of all ETF fund flow according to morningstar \cite{monring_star_ESG}.

To address the impact of ESG factors on stock preformance, this paper is broken into five sections. \textit{Section \ref{Literature}} will provide a  brief literature review of papers looking at abnormal performance around ESG criteria as well as discussion of asset pricing models to capture ESG factors in performance. Next \textit{ Section \ref{Data}}  will provide a brief discussion of the data used in this paper as well as the data cleaning process and summary statistics. Following the Data, \textit{Section \ref{Methodology}} will breakdown of the methodology for the paper. Finally, \textit{Section \ref{Results}}  will summarize the results of the regressions, \textit{Section \ref{Robustness}} will check the robustness of the results and \textit{Section \ref{Conclusion}} will provide a conclusion and possible extensions for next steps. 

\section{Literature Review}
This section will be broke into two components, we will first look at theoretical models for asset purchasing and then we will introduce some of the mythology for introducing a EGS factor into the asset pricing framework.
to account for the ESG factor of a five factor famma french model, the authers used the S\&P ESG index as the high ESG index and the S\&P petroleum index as the low index creating a facotr by subtracting the retunrs from the high ESG portfolio from the low ESG portfolio. \cite{Gregory2020TheGP}
\label{Literature}

\section{Data}
\label{Data}
The data used for this paper was soured three databases. The daily performance of the individual stock prices was collected from CRSP's primarily daily stock prices. Data on the ESG portfolio and well as the petroleum portfolio was extracted from compustat as the daily average returns for MSCI World ESG Focus Index and S/&P petroleum index. Finally, daily Fama French factors for SMB, CMA,HML, and RMW were sourced from the Kenneth R. French - Data Library.  An ESG factor, demoted as EMP (ESG minus petroleum), was created by taking the  MSCI World ESG Focus Index returns and subtracting them from the  S\&P petroleum index. Petroleum companies act as a stand in for the opposite of ESG as the  petroleum space is particularly inept to adapt to ESG  changes, a fact well documented in academic literature \cite{Frynas2005TheFD}. In addition, in section \ref{Robustness} for robustness I will repeat the analysis replacing the  S\&P petroleum index with S\&P Metals & Mining Select Industry Index as the opposed portfolio for the EMP factor. It has been also well documented that the mining industry also struggles to adapt to ESG factors due to the nature of the industry \cite{Kapelus2002MiningCS}.
 
 

 
 Data was collected between June 2011 and March 2021. The started was determined by the initiation of the MSCI ESG index. With the selected window, CRSP data for daily returns was collected on each available company. Daily returns for the individual companies as well as each of the Fama factors and the EMP factor was collapsed by month by suming the daily changes in each given month and year. The collapses resulted in 830,889 observation for 12,194 unique stocks which gives an average of 68 months of data for each unique security.  Outlined in table are  some of the summary statistics for the raw data.
 
 \begin{center}
     \textbf{Table 1}\\
 {
\def\sym#1{\ifmmode^{#1}\else\(^{#1}\)\fi}
\begin{tabular}{l*{1}{cccc}}
\hline\hline
                    &        mean&          sd&         min&         max\\
\hline
Excess Returns      &       0.008&       0.152&      -3.701&      14.541\\
Market              &       0.009&       0.042&      -0.146&       0.131\\
HML                 &      -0.004&       0.027&      -0.176&       0.094\\
SMB                 &      -0.001&       0.026&      -0.160&       0.077\\
RMW                 &       0.000&       0.015&      -0.043&       0.049\\
CMA                 &      -0.001&       0.014&      -0.036&       0.045\\
EMP                 &      -0.007&       0.090&      -0.495&       0.361\\
Market Cap          & 4299884.523&23004970.838&      82.632&   2.164e+09\\
\hline
Observations        &      830889&            &            &            \\
\hline\hline
\end{tabular}
}

 \end{center}
 
\section{Methodology}
\label{Methodology}
As discussed in the literature review there are numerous different models for assessing asset pricing and abnormal returns. In this paper I will be using an extension on on Fama French's 2015 paper ``A five factor asset pricing model". This method of analysis is builds off of the famma french three factor model and relies on the 2 pass sort to assign a series of beta decile and size decile based portfolios. Breaking the portfolios into these distinct groups allows for further analysis of the effect across these two factors do asses the abnormal returns. 

To determine the beta deciles for individual securities first a beta must be calculate for each security. To calculate the betas, a  simple regression outlined in equation \eqref{prerank_reg} was preformed. In this regression, a vector of  excess market returns in the current period and a vector of market returns from the previous period is regressed on a vector of current periods individual stock excess returns (i.e $R_{i,t}-R_{rf,t}$). The time movements for the data moved in month increments and the length of the vector ranged for 12 month to 60 months depending on the availability of data. Next, the beta values are summed together to get the pre-ranking beta value. 

\begin{equation}
    R_{i,t}^e = a_{i,t} + \beta_{i1,t}(R_{m,t}-R_{rf,t})+\beta_{i2,t}(R_{m,t-1}-R_{rf,t-1})+\epsilon_{i,t}
    \label{prerank_reg}
\end{equation}
\begin{equation}
    \beta_{sum} = \sum_{k=1}^2\beta_{ik,t}
\end{equation}


Before assigning the deciles a data cleaning process was applied. A small portion of prerankning beta's are far larger than would be expected, as such these vales were removed from the sample by removing top and bottom 4\% of companies by preranking beta. A similar data cleaning process was also conducted in the Fama french five factor model. In addition to removing the very large and very small preranking beta values I also removed the companies with market capitalization less than 10M dollars in market capitalization. After sorting by the beta decile's size deciles were asigned based on the CRSP NYSE/NYSE American/NASDAQ market capitalization.


*dicuss how we collapse the prolfio from 10 to 2 size

Decile-sorted indexes based on
market capitalization
TO find the beta deciles 


To capture abnormal returns we will consider 5 factors that is the 5 factors from the famma french paper small minus big(SMB),CMA, HML, and market portfolio(I.e model, consisting of market, size, value, profitability, and investment factors;). In addition I will ad another term to address how stocks move with the movement to a ESG porlfio. For this purpose we will use the MSCI ESG all world index. 

\begin{equation}
\begin{split}
    R_{it}^e = a_i+b_i(R_{Mt}-R_{fr})+s_iSMB_t+ & h_iHML_t+ & r_iRMW_t+c_iCMA_t+e_iEMP_t
    \end{split}
    \label{six_factor_basic_reg}
\end{equation}
In equation \eqref{six_factor_basic_reg} $HML_{t}$ is the difference between the returns on diversified portfolios of high and low book to market stocks, $SMB_{t}$ i the return on a diversified portfolio of small stocks minus the return on a diversified portfolio of big stocks, $RMW_t$ is the difference between the returns on diversified portfolios of stocks with robust and weak profitability,$CMA_{t}$ is the difference between the returns on diversified portfolios of the stocks of low and high investment firms,$EMP_t$ is the difference between MSCI World ESG Focus Index returns and the S\&P petroleum index. finally, $R^e_{it}$ is the return of the individual secutiy minus the risk freee rate ($R_{it}-R_{rf}$)


\section{Results}
\label{Results}

\section{Robustness Checks}
 %as dicussed in section \ref{Data} [talk about using mining instead of petrolum ]
\label{Robustness}

\section{Conclusion}
\label{Conclusion}

\bibliography{Matt_Bib.bib}{}
\bibliographystyle{plain}
%\nocite{*}
%\printbibliography

\section{Code}
\textbf{Data Cleaning and setting up for Pre-Ranking}
\lstinputlisting[language=Stata]{Code/code_for_returns.do}
\textbf{Pre-Ranking Beta}
\lstinputlisting{Code/stage_1_pre_rank.m}

\end{document}
