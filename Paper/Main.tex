\documentclass[12pt,oneside,reqno]{amsart}
\usepackage{mathtools, stackengine}
\numberwithin{equation}{section}
\usepackage{esint}
%\usepackage{undertilde}
\newcommand\norm[1]{\left\lVert#1\right\rVert}
\usepackage{tikz}
\usetikzlibrary{arrows} 
\usetikzlibrary{calc,angles,quotes}
\newtheorem{theorem}{Theorem}[section]
\newtheorem{remark}[theorem]{Remark}
\newtheorem{lemma}[theorem]{Lemma}
\newtheorem{corollary}[theorem]{Corollary}
\usepackage{caption}
\usepackage{subcaption}
\usepackage{longtable}
\usepackage{color,soul}
\usepackage{bm}
\usepackage{setspace}
\newcommand{\ba}{\backslash}
\doublespacing
\usepackage{afterpage}
\usepackage{listings}
\usepackage{geometry} %may be sus
\usepackage[framed,numbered]{matlab-prettifier}
% turn "backspace" character (ASCII 8) into a tab (9) to avoid Matlab warning backspace
\catcode8=9
\lstset{
  style             = Matlab-editor,
  basicstyle        = {\ttfamily\tiny},
  mlshowsectionrules = true,
  upquote           = true
}

\newcommand{\spacer}{\vspace{6mm} \noindent}
\usepackage{parskip}% http://ctan.org/pkg/parskip

\usepackage{enumitem}

% BIB stuff

%\usepackage{biblatex}
%\addbibresource{Matt_Bib.bib}

\usepackage{hyperref}



\definecolor{codeblue}{rgb}{0.29296875, 0.51953125, 0.68359375}
\definecolor{codegreen}{rgb}{0.47265625, 0.62890625, 0.40234375}
\definecolor{codegray}{rgb}{0.95703125, 0.95703125, 0.95703125}
\definecolor{codecrimson}{rgb}{0.87109375,0.3984375,0.3984375}



\lstdefinelanguage{Stata}{
    % System commands
    morekeywords=[1]{regress, reg, summarize, sum, display, di, generate, gen, bysort, use, import, delimited, predict, quietly, probit, margins, test},
    % Reserved words
    morekeywords=[2]{aggregate, array, boolean, break, byte, case, catch, class, colvector, complex, const, continue, default, delegate, delete, do, double, else, eltypedef, end, enum, explicit, export, external, float, for, friend, function, global, goto, if, inline, int, local, long, mata, matrix, namespace, new, numeric, NULL, operator, orgtypedef, pointer, polymorphic, pragma, private, protected, public, quad, real, return, rowvector, scalar, short, signed, static, strL, string, struct, super, switch, template, this, throw, transmorphic, try, typedef, typename, union, unsigned, using, vector, version, virtual, void, volatile, while,},
    % Keywords
    morekeywords=[3]{forvalues, foreach, set},
    % Date and time functions
    morekeywords=[4]{bofd, Cdhms, Chms, Clock, clock, Cmdyhms, Cofc, cofC, Cofd, cofd, daily, date, day, dhms, dofb, dofC, dofc, dofh, dofm, dofq, dofw, dofy, dow, doy, halfyear, halfyearly, hh, hhC, hms, hofd, hours, mdy, mdyhms, minutes, mm, mmC, mofd, month, monthly, msofhours, msofminutes, msofseconds, qofd, quarter, quarterly, seconds, ss, ssC, tC, tc, td, th, tm, tq, tw, week, weekly, wofd, year, yearly, yh, ym, yofd, yq, yw,},
    % Mathematical functions
    morekeywords=[5]{abs, ceil, cloglog, comb, digamma, exp, expm1, floor, int, invcloglog, invlogit, ln, ln1m, ln, ln1p, ln, lnfactorial, lngamma, log, log10, log1m, log1p, logit, max, min, mod, reldif, round, sign, sqrt, sum, trigamma, trunc,},
    % Matrix functions
    morekeywords=[6]{cholesky, coleqnumb, colnfreeparms, colnumb, colsof, corr, det, diag, diag0cnt, el, get, hadamard, I, inv, invsym, issymmetric, J, matmissing, matuniform, mreldif, nullmat, roweqnumb, rownfreeparms, rownumb, rowsof, sweep, trace, vec, vecdiag, },
    % Programming functions
    morekeywords=[7]{autocode, byteorder, c, _caller, chop, abs, clip, cond, e, fileexists, fileread, filereaderror, filewrite, float, fmtwidth, has_eprop, inlist, inrange, irecode, matrix, maxbyte, maxdouble, maxfloat, maxint, maxlong, mi, minbyte, mindouble, minfloat, minint, minlong, missing, r, recode, replay, return, s, scalar, smallestdouble,},
    % Random-number functions
    morekeywords=[8]{rbeta, rbinomial, rcauchy, rchi2, rexponential, rgamma, rhypergeometric, rigaussian, rlaplace, rlogistic, rnbinomial, rnormal, rpoisson, rt, runiform, runiformint, rweibull, rweibullph,},
    % Selecting time-span functions
    morekeywords=[9]{tin, twithin,},
    % Statistical functions
    morekeywords=[10]{betaden, binomial, binomialp, binomialtail, binormal, cauchy, cauchyden, cauchytail, chi2, chi2den, chi2tail, dgammapda, dgammapdada, dgammapdadx, dgammapdx, dgammapdxdx, dunnettprob, exponential, exponentialden, exponentialtail, F, Fden, Ftail, gammaden, gammap, gammaptail, hypergeometric, hypergeometricp, ibeta, ibetatail, igaussian, igaussianden, igaussiantail, invbinomial, invbinomialtail, invcauchy, invcauchytail, invchi2, invchi2tail, invdunnettprob, invexponential, invexponentialtail, invF, invFtail, invgammap, invgammaptail, invibeta, invibetatail, invigaussian, invigaussiantail, invlaplace, invlaplacetail, invlogistic, invlogistictail, invnbinomial, invnbinomialtail, invnchi2, invnF, invnFtail, invnibeta, invnormal, invnt, invnttail, invpoisson, invpoissontail, invt, invttail, invtukeyprob, invweibull, invweibullph, invweibullphtail, invweibulltail, laplace, laplaceden, laplacetail, lncauchyden, lnigammaden, lnigaussianden, lniwishartden, lnlaplaceden, lnmvnormalden, lnnormal, lnnormalden, lnwishartden, logistic, logisticden, logistictail, nbetaden, nbinomial, nbinomialp, nbinomialtail, nchi2, nchi2den, nchi2tail, nF, nFden, nFtail, nibeta, normal, normalden, npnchi2, npnF, npnt, nt, ntden, nttail, poisson, poissonp, poissontail, t, tden, ttail, tukeyprob, weibull, weibullden, weibullph, weibullphden, weibullphtail, weibulltail,},
    % String functions 
    morekeywords=[11]{abbrev, char, collatorlocale, collatorversion, indexnot, plural, plural, real, regexm, regexr, regexs, soundex, soundex_nara, strcat, strdup, string, strofreal, string, strofreal, stritrim, strlen, strlower, strltrim, strmatch, strofreal, strofreal, strpos, strproper, strreverse, strrpos, strrtrim, strtoname, strtrim, strupper, subinstr, subinword, substr, tobytes, uchar, udstrlen, udsubstr, uisdigit, uisletter, ustrcompare, ustrcompareex, ustrfix, ustrfrom, ustrinvalidcnt, ustrleft, ustrlen, ustrlower, ustrltrim, ustrnormalize, ustrpos, ustrregexm, ustrregexra, ustrregexrf, ustrregexs, ustrreverse, ustrright, ustrrpos, ustrrtrim, ustrsortkey, ustrsortkeyex, ustrtitle, ustrto, ustrtohex, ustrtoname, ustrtrim, ustrunescape, ustrupper, ustrword, ustrwordcount, usubinstr, usubstr, word, wordbreaklocale, worcount,},
    % Trig functions
    morekeywords=[12]{acos, acosh, asin, asinh, atan, atanh, cos, cosh, sin, sinh, tan, tanh,},
    morecomment=[l]{//},
    % morecomment=[l]{*},  // `*` maybe used as multiply operator. So use `//` as line comment.
    morecomment=[s]{/*}{*/},
    % The following is used by macros, like `lags'.
    morestring=[b]{`}{'},
    % morestring=[d]{'},
    morestring=[b]",
    morestring=[d]",
    % morestring=[d]{\\`},
    % morestring=[b]{'},
    sensitive=true,
}


\begin{document}
\section{Introduction and Motivation}
Over the last decade investors have become increasingly concerned with extra factors outside of financials to address the performance of stocks. Research from Mckendsey has shown that there is a direct correlation between diversity inside of executive team leadership and company performance. A rising threat of global climate related issues and workplace culture to determine top talent have all become extremely important for company management to consider on the basis of long term performance.  


this paper will be broken up into five  sections. \textit{Section \ref{Literature}} will provide a  brief literature review of papers looking abnormal performance around ESG criteria as well as discussion of possible asset pricing models to capture ESG factors in performance. Next in  \textit{ Section \ref{Data}}  I will discuss the data used in this papers as well as the data cleaning process and summary statistics. Following the Data, \textit{Section \ref{Methodology}} will breakdown of the methodology for the paper. Finally, \textit{Section \ref{Results}}  will summarize the results of the regressions, \textit{Section \ref{Robustness}} will check the robustness of the results and \textit{Section \ref{Conclusion}} will provide a conclusion and possible extensions for next steps. 

\section{Literature Review}
\label{Literature}

\section{Data}
\label{Data}
The data used for this paper was soured three databases. The daily performance of the individual stock prices was collected from CRSP's primarily daily stock prices. Data on the ESG portfolio was extracted from compustat as the daily average returns for MSCI World ESG Focus Index. Finally, daily Fama French factors for SMB, CMA,HML, and RMW were sourced from the Kenneth R. French - Data Library.  The time period for the data for the paper is between June 2011 and December 2020. The segment of stock data is determined by the length of available performance for the MSCI World ESG Focus Index as this index was initiated in June 2011. Daily values were summed and collapsed to created monthly values for factor and individual security performance. 

we looking 830,891 observation for  12,194 unique stocks which gives an average of 68months of data 
\section{Methodology}
\label{Methodology}
As dicussed in the literature review there are numerous different models for assessing asset pricing and abnormal returns. In this paper I will be using an extension on on Fama French's 2015 paper ``A five factor asset pricing model".

To capture abnormal returns we will consider 5 factors that is the 5 factors from the famma french paper small minus big(SMB),CMA, HML, and market portfolio(I.e model, consisting of market, size, value, profitability, and investment factors;). In addition I will ad another term to address how stocks move with the movement to a ESG porlfio. For this purpose we will use the MSCI ESG all world index. 

\begin{equation}
\begin{split}
    R_{it}^e = a_i+b_i(R_{Mt}-R_{fr})+s_iSMB_t+ & h_iHML_t+ & r_iRMW_t+c_iCMA+e_i(R_{ESG t}-R_{rf})
    \end{split}
    \label{six_factor_basic_reg}
\end{equation}
In equation \eqref{six_factor_basic_reg} $HML_{t}$ is the difference between the returns on diversified portfolios of high and low book to market stocks, $SMB_{t}$ i the return on a diversified portfolio of small stocks minus the return on a diversified portfolio of big stocks, $RMW_t$ is the difference between the returns on diversified portfolios of stocks with robust and weak profitability,$CMA_{t}$ is the difference between the returns on diversified portfolios of the stocks of low and high investment firms, $e^e_{it}$ is the return of the individual secutiy minus the risk freee rate ($R_{it}-R_{rf}$)


\section{Results}
\label{Results}

\section{Robustness Checks}
\label{Robustness}

\section{Conclusion}
\label{Conclusion}
\section{Code}
\textbf{Data Cleaning and setting up for Pre-Ranking}
\lstinputlisting[language=Stata]{Code/code_for_returns.do}
\textbf{Pre-Ranking Beta}
\lstinputlisting{Code/stage_1_pre_rank.m}

\end{document}
