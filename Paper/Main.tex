\documentclass[12pt,oneside,reqno]{amsart}


\begin{document}
Over the last decade investors have become increasingly concerned with extra factors outside of financials to address the performance of stocks. Research from Mckendsey has shown that there is a direct correlation between diversity inside of executive team leadership and company performance. A rising threat of global climate related issues and workplace culture to determine top talent have all become extremely important for company management to consider on the basis of long term performance.  


this paper will be broken up into four major sections. We will first asses a brief literature review of papers looking abnormal performance around ESG criteria as well as discussion of possible models to apply asset pricing to ESG factors. Next I will discuss the data used in this papers as well as the data cleaning process and summary statistics. Following the Data will be a breakdown of the methodology for the paper. The final two sections will provide the results of the regressions and a conclusion and possible next steps. 
\section{Methodology}
As dicussed in the literature review there are numerous different models for assessing asset pricing and abnormal returns. In this paper I will be using an extension on on Fama French's 2015 paper ``A five factor asset pricing model".

To capture abnormal returns we will consider 5 factors that is the 5 factors from the famma french paper small minus big(SMB),CMA, HML, and market portfolio(I.e model, consisting of market, size, value, profitability, and investment factors;). In addition I will ad another term to address how stocks move with the movement to a ESG porlfio. For this purpose we will use the MSCI ESG all world index. 

\begin{equation}
\begin{split}
    R_{it}^e = a_i+b_i(R_{Mt}-R_{Fr})+s_iSMB_t+ & h_iHML_t+ & r_iRMW_t+c_iCMA+e_i(R_{ESG t}-R_{rf})
    \end{split}
\end{equation}
where b
\end{document}