Over the last decade, investors have become increasingly interested in monitoring factors beyond traditional financial metrics to find opportunities to achieve excess returns. In this pursuit,  many investors have focused their attention on the ESG, Environmental Social Governance, framework. Using the ESG framework, investors evaluate a firm's sustainability and societal impact, which are typically overlooked in traditional asset pricing models. This method of investing has garnered quick traction as global climate change threatens the longevity of critical global infustructure, and diversity initiatives shape the modern workforce  \cite{holder_three-degree_nodate, mckinsey_diversity}. A study conducted by Morgan Stanley found in 2016, just 10\% of surveyed asset managers had adopted sustainable investing, a number that quickly grew to over 70\% by 2018 \cite{morgan_stanley_sustainable_2019}. 

This new wave of investor sentiment has created strong tailwinds for sustainable investment products. In 2020 alone, fund flows into sustainable funds grew by \$51 billion  which is up over 100\% from the 2019 fund flow into sustainable investment products of about \$21 billion, placing total assets under management in sustainable products at over \$250 billion. Fund flows into sustainable products for 2020 accounted for a little less a quarter of all US mutual fund flow and about one-fifteenth of all ETF fund flow according to Morningstar \cite{monring_star_ESG}.

In this paper, I assess the impact of ESG investing on the excess returns through a five-factor Fama French framework. I establish two different methods for defining ESG risk premia and find dissimilar results. When defining ESG risk premia as the difference between the MSCI ESG index and the S\&P petroleum index, I find a robust risk premia, which suggests that a one-unit change in beta results in an 8.7\% increase in excess returns for size decile portfolios. However, when defining ESG risk premia as the difference between high ESG firms minus low ESG firms as rated by MSCI, there are no statistically measurable risk premia. Further, when analyzing the risk premia at the one-digit SIC level, only one SIC code yields a statistical relationship between excess returns and the ESG factor. 
