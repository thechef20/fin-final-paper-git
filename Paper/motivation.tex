Over the last decade, investors have become increasingly interested in factors outside of traditional financial metrics when assessing the possibility of an investment's returns. Much of this enthusiasm has been based on an ESG, Environmental Social Governance, framework. ESG investing considers how firms respond to these factors as potential impetuses for excess performance. This increased attention is especially visible in Morgan Stanley's 2019 survey of the ESG landscape. Morgan Stanley  found that 75\% of assessment managers say that "their firm has adopted sustainable investing." Only two years prior, in 2016, just 10\% of surveyed asset managed had adopted sustainable investing, demonstrating the rapid speed of adoption for ESG investing analysis \cite{morgan_stanley_sustainable_2019}. Practitioners have utilized the ESG framework to reduce risk in their investments while also finding abnormal returns.

Global climate change presents a host of possible and sizable financial risks from increased water levels, which could threaten major global cities' safety, including Miami, Hong Kong, and Shanghai, with the potential to displace 100's of millions of people \cite{holder_three-degree_nodate}. Global climate change is also shifting the landscape for global agricultural production as more extreme weather conditions, including droughts, pose problems to the most fundamental building block of economic activity. In addition to risk mitigation, investors look to utilize  ESG principles for superior returns. A research report published by McKinsey and Company linked gender diversity on executive teams to higher profitability, and superior value creation \cite{mckinsey_diversity}.

This new wave of investor sentiment has created strong tailwinds for sustainable investment projects. In 2020 alone, fund flows into sustainable funds grew by 51 billion dollars which is up over 100\% for the 2019 fund flow into sustainable investment products of about 21, placing total assets under management in sustainable products at over 250B. Fund flows into sustainable products for 2020 accounted for a little less a quarter of all US mutual fund flow and about one-fifteenth of all ETF fund flow according to morningstar \cite{monring_star_ESG}.

In this paper, I assess the impact of ESG investing on the excess market through a five-factor Fama McBeth framework. I establish two different methods for assessing ESG risk premia and find dissimilar results. When defining ESG risk premia as the difference between the MSCI ESG index and the S\&P Petroleum index, I find a robust risk premia, which suggests that a one-unit change in beta results in an 8.7\% increase in excess returns for size decile portfolios. However, when defining ESG risk premia as the difference between high ESG firms minus low ESG firms as rated by MSCI, there are no statistically measurable risk premia. Further, when analyzing the risk premia at the one-digit SIC level, there is still no statistical relationship between SIC industry and excess returns and the ESG factor. 

I find that the definition of ESG in market return results in vastly different outcomes. When building a market index 
this paper is broken into five sections. \textit{Section \ref{Literature}} will provide a  brief literature review of papers looking at abnormal performance around ESG criteria as well as discussion of asset pricing models to capture ESG factors in performance. Next \textit{ Section \ref{Data}}  will provide a brief discussion of the data used in this paper as well as the data cleaning process and summary statistics. Following the Data, \textit{Section \ref{Methodology}} will breakdown of the methodology for the paper. Finally, \textit{Section \ref{Results}}  will summarize the results of the regressions, \textit{Section \ref{Robustness}} will check the robustness of the results and \textit{Section \ref{Conclusion}} will provide a conclusion and possible extensions for next steps. 