As discussed in the literature review, there are numerous models for assessing asset pricing and excess returns. In this paper, I will be using an extension of Fama French's 2015 paper ``A five-factor asset pricing model". This analysis method is built on the Fama French three-factor model and relies on the two-pass sort. First, I will explain the methodology for assessing the sensitivity and risk premium for the ESG factor created from the MSCI ESG index minus the petroleum index. Next, I will discuss the methodology for MSCI ESG rating ESG factors. I will conclude with a brief description of the newey west errors and the setup of assessing statistical significance. 


For the first part of this study, I sort each security into decile groups based on beta and size. The size and beta deciles ratings are provided by CRSP and are merged on the individual firm level for each month of data. Next, I conducted the first pass for the two-pass sort by running the regression of excess returns on market risk premia, the four Fama French factors, and the EMP factor as described in equation \eqref{six_factor_basic_reg}. This regression was run twice: once on the beta-ranked portfolios and again on the size-ranked portfolios. The first pass provides insight into the sensitivity of the factor on the declie index. 
\begin{equation}
\begin{split}
    R_{i,t}^e = a_i+b_i(R_{M,t}-R_{fr,t})+s_iSMB_t+ h_iHML_t+  r_iRMW_t+c_iCMA_t+e_iEMP_t +\epsilon_{i,t}
    \end{split}
    \label{six_factor_basic_reg}
\end{equation}
\begin{center}
Where:\\
    $R_{i,t}^e = R_{it}-R_{rf,t}$
\end{center}
Next, I used the output data from equation \eqref{six_factor_basic_reg} to run the second regression in the two-pass sort as described in the equation  \eqref{second_pass}. The second pass represents the risk premia for each of the six factors in the model. This second pass was performed for both beta and size-based portfolios.
\begin{equation}
    E[R^e]= b\lambda_1+s\lambda_2 +h\lambda_3 +r\lambda_4 +c \lambda_4 + e\lambda_5
    \label{second_pass}
\end{equation}\\
The second set of regressions is focused on establishing a link between MSCI ESG ratings and excess returns. Similar to the first set of regressions, I use a two-pass sort methodology using both size and beta ranked portfolios. To create the ESG factor for this set of analyses, I used the ESG factor described in Section \ref{Data}. The process of implementing the regression follows the same basic form as the one for the MSCI ESG index and petroleum index ESG factor by first running a pass on the portfolio by size(beta) decile and following that with a cross-sectional analysis to quantify the risk premium. 

Finally, in determining the statistical significance of each of the coefficients, it is important to discuss the newly west error. The newly west error is a replacement for standard error in a traditional OLS regression. The advantage of the newly west is an ability to account for serial correlation within the model. To view any of the supporting code for the analysis, view the link in the citation \cite{Matt_code}.
