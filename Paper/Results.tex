Table 2 represents a matrix of values for the first pass in the Fama French two-pass sort, described in equation \eqref{six_factor_basic_reg}. The row axis is the distinct size deciles, where 1 is the smallest 10\% of companies by market cap companies, and 10 is the decile with the largest 10\% of companies by market cap. The column axis includes estimated coefficients for the intercept, five Fama French factors, and the EMP factor. $HML_{t}$ is the difference between the returns on diversified portfolios of the high and low book to market stocks.  $SMB_{t}$ is the return on a diversified portfolio of small stocks minus the return on a diversified portfolio of big stocks. $RMW_t$ is the difference between the returns on diversified portfolios of stocks with robust and weak profitability. $CMA_{t}$ is the difference between the returns on diversified portfolios of the stocks of low and high investment firms. $EMP_t$ is the difference between MSCI World ESG Focus Index returns and the S\&P petroleum index. Finally, the intercept is the excess returns for the given decile. The results in this table demonstrate that for HML, CMA, and EMP factors, there is a size-based relationship wherein the larger the size decile, the smaller the sensitivity effect is on the factor. In addition, the EMP factor also exhibits a consistent negative relationship between the EGS factor and returns. This relationship indicates that investors are returning a premium by investing in ESG companies over the contra ESG counterparts and that this relationship is largest for small companies. The results from this statistical test are significant at the 5\% level for the EMP factor for all of the portfolios except the smallest and largest size portfolios. However, several of the Fama French factors are also not significant for the largest and smallest portfolios, which may indicate that these tail portfolios might be behaving differently due to their composition rather than a reflection of the factor. 


\begin{center}
    \paperspacingnarrow
    {
\def\sym#1{\ifmmode^{#1}\else\(^{#1}\)\fi}
\begin{tabular}{l*{1}{cccccccc}}
\hline\hline
            &\multicolumn{1}{c}{(1)}&            &            &            &            &            &            &            \\
            &\_b\_market\_minus\_rf&      \_b\_hml&      \_b\_smb&      \_b\_rmw&      \_b\_cma&      \_b\_emp&         \_R2&portfolio\_number\\
\hline
1           &       1.009&       -.574&        .555&       -.235&        .946&        .126&        .547&           1\\
2           &        .781&       -.192&        .365&       -.131&         .39&        .172&        .723&           2\\
3           &        .881&       -.043&        .259&       -.025&        .251&        .123&        .863&           3\\
4           &        .859&        .095&        .423&       -.088&        .374&        .113&        .901&           4\\
5           &        .884&        -.03&        .563&        .088&        .375&        .097&        .919&           5\\
6           &       1.015&        .064&        .597&        .015&        .249&        .075&        .945&           6\\
7           &       1.132&        .113&        .699&        .146&        .252&        .067&         .96&           7\\
8           &        1.11&        .056&        .584&        .028&        .293&        .046&        .969&           8\\
9           &       1.078&        .139&        .381&        .052&        .049&        .037&        .958&           9\\
10          &        .991&        .061&        .062&       -.025&        .164&        .017&        .968&          10\\
Total       &        .974&      -.0311&       .4488&      -.0175&       .3343&       .0873&       .8753&         5.5\\
\hline\hline
\end{tabular}
}
\\
    \textbf{Table 2:} represents the first pass of the two pass sort where the portfolios are broken up by size. the row axis is portfolio and the column axis is the  market beta,four Fama French factors and the EMP factor\\
    \paperspacingwide
\end{center}

Table 3 demonstrates the risk premia for each of the factors using the second step in the two-pass sort for Size decile portfolios. The market factor, the SMB factor, the CMA factor, and the EMP factor are all positive and statistically significant at the 1\% level. The other two factors have a negative sign and are not statistically significant at the 10\% level.  EMP factor demonstrates that a 1 unit change in beta corresponds with an 8.7\% increase in returns. Although this is by far the smallest coefficient of the four statistically significant returns, this term is still economically meaningful and represents a risk premia for investing in high ESG firms. Overall the model appears to be a fairly good fit with a $R^2$ value of 87.5\%, which is lower than some previous literature using the petroleum index but still indicates a well fit model \hl{cite this paper} 


\begin{center}
    \paperspacingnarrow
    \begin{tabular}{lcc} \hline
 & (1) & (2) \\
VARIABLES & Size Portoflios & Beta Portoflios \\ \hline
 &  &  \\
market\_minus\_rf & 0.974*** & 0.996*** \\
 & (0.0468) & (0.156) \\
hml & -0.0310 & 0.0254 \\
 & (0.0785) & (0.0533) \\
smb & 0.449*** & 0.425*** \\
 & (0.0699) & (0.0972) \\
rmw & -0.0177 & -0.00593 \\
 & (0.0406) & (0.0982) \\
cma & 0.334*** & 0.281** \\
 & (0.0822) & (0.0998) \\
esg\_minus\_rf & -0.0871*** & -0.0736** \\
 & (0.0195) & (0.0250) \\
Constant & -9.47e-05 & -0.000961 \\
 & (0.00210) & (0.000824) \\
 &  &  \\
Observations & 1,150 & 1,150 \\
R-squared & 0.875 & 0.860 \\
 Number of groups & 10 & 10 \\ \hline
\multicolumn{3}{c}{ Standard errors in parentheses} \\
\multicolumn{3}{c}{ *** p$<$0.01, ** p$<$0.05, * p$<$0.1} \\
\end{tabular}
\\
    \textbf{Table 3:} second pass results for the size and beta decile portfolios\\
    \paperspacingwide
\end{center}

Table 4 details the relationship outlined in equation \eqref{six_factor_basic_reg} in the methodology where the row axis is the beta decile. In this table,  the beta decile 1 comprises the companies with the lowest market beta's. Similarly, beta decile 10 is the portfolio of companies with the largest market beta values.  Unlike the size decile portfolios, the beta decile portfolios do not have consistently negative relations with respect to decile. Rather the relationship appears to have a dome shape where portfolio 6 has the smallest positive sensitivity to the EMP factor, and larger or smaller beta portfolios appear to have a larger sensitivity to the  EMP factor. \hl{why might this be}. Similar to the size decile analysis, the EMP coefficients are all very significant, with every portfolio having significance at the 5\% level. 

When considering the second phase of the two-pass sort for the beta decile portfolios, the similarities to the size decile continue wherein market returns, SMB CMA and EMP exhibit positive risk premia. The other two factors remain statistically insignificant, and the MHL factor changed signs and now has a positive relationship. The EMP coefficient indicates that a one-unit change in beta would result in a 7.36\% change in returns for a high ESG company. This result is slightly smaller than the result from the size portfolio but still demonstrates a statistical relationship in the same order of magnitude. The model is also a good fit with an $R^2$ value of 86.0% 

\begin{center}
    \paperspacingnarrow
    {
\def\sym#1{\ifmmode^{#1}\else\(^{#1}\)\fi}
\begin{tabular}{l*{1}{cccccccc}}
\hline\hline
            &\multicolumn{1}{c}{(1)}&            &            &            &            &            &            &            \\
            &           .&            &            &            &            &            &            &            \\
            &\_b\_market\_minus\_rf&      \_b\_hml&      \_b\_smb&      \_b\_rmw&      \_b\_cma&\_b\_esg\_minus\_rf&         \_R2&portoflio\_numer\\
\hline
1           &       1.009&       -.574&        .555&       -.235&        .946&       -.126&        .547&           1\\
2           &        .781&       -.192&        .365&       -.131&         .39&       -.172&        .723&           2\\
3           &        .881&       -.043&        .259&       -.025&        .251&       -.123&        .863&           3\\
4           &        .859&        .095&        .423&       -.088&        .374&       -.113&        .901&           4\\
5           &        .884&        -.03&        .563&        .088&        .375&       -.097&        .919&           5\\
6           &       1.015&        .064&        .597&        .015&        .249&       -.075&        .945&           6\\
7           &       1.132&        .113&        .699&        .146&        .252&       -.067&         .96&           7\\
8           &        1.11&        .056&        .584&        .028&        .293&       -.046&        .969&           8\\
9           &       1.078&        .139&        .381&        .052&        .049&       -.037&        .958&           9\\
10          &        .991&        .061&        .062&       -.025&        .164&       -.017&        .968&          10\\
Total       &        .974&      -.0311&       .4488&      -.0175&       .3343&      -.0873&       .8753&         5.5\\
\hline
\(N\)       &          10&            &            &            &            &            &            &            \\
\hline\hline
\end{tabular}
}
\\
    \textbf{Table 4:} Beta decile by estimated coefficients for the  five fama french factors and EMP factor\\
    \paperspacingwide
\end{center}

Moving on to the MSCI based portfolio, a two-pass sort was also used to assess the sensitivity and risk premia on the ESG portfolio on both size and  beta porflio 

\begin{center}
    \paperspacingnarrow
\begin{tabular}{lcc} \hline
 & (1) & (2) \\
VARIABLES & Size portfolio & Beta Portfolio \\ \hline
 &  &  \\
Market & 0.985*** & 1.012*** \\
 & (0.0442) & (0.165) \\
HML & 0.0837 & 0.123 \\
 & (0.0641) & (0.0710) \\
SMB & 0.481*** & 0.455*** \\
 & (0.0721) & (0.105) \\
RMW & 0.00197 & -0.00415 \\
 & (0.0228) & (0.0917) \\
CMA & 0.200** & 0.173** \\
 & (0.0638) & (0.0699) \\
ESG & 0.00474 & 0.0145 \\
 & (0.0142) & (0.0132) \\
Constant & -0.000769 & -0.00163 \\
 & (0.00211) & (0.00110) \\
 &  &  \\
Observations & 1,150 & 1,150 \\
R-squared & 0.851 & 0.837 \\
 Number of groups & 10 & 10 \\ \hline
\multicolumn{3}{c}{ Standard errors in parentheses} \\
\multicolumn{3}{c}{ *** p$<$0.01, ** p$<$0.05, * p$<$0.1} \\
\end{tabular}
\\
\textbf{Table 7:} SIC top 10\% minus bottom 10\%
    \paperspacingwide
\end{center}