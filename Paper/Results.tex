Table 3 represents a matrix of values for the first pass in the Fama French two-pass sort, described in equation \eqref{six_factor_basic_reg}. The column axis is the distinct size deciles where 1 is the smallest market cap companies, and 10 is the decline with the largest market cap companies. The row axis includes estimated coefficients for the intercept, five Fama French factors, and the EMP factor. $HML_{t}$ is the difference between the returns on diversified portfolios of the high and low book to market stocks.  $SMB_{t}$ is the return on a diversified portfolio of small stocks minus the return on a diversified portfolio of big stocks. $RMW_t$ is the difference between the returns on diversified portfolios of stocks with robust and weak profitability. $CMA_{t}$ is the difference between the returns on diversified portfolios of the stocks of low and high investment firms. $EMP_t$ is the difference between MSCI World ESG Focus Index returns and the S\&P petroleum index. Finally, the intercept is the excess returns for the given decile. The results in this table demonstrate that for HML, CMA, and EMP factors, there is a size-based relationship wherein the larger the size decile, the smaller the sensitivity effect is on the factor. In addition, the EMP factor also exhibits a consistent negative relationship between the EGS factor and returns. This relationship indicates that investors are actually paying a premium to invest in ESG companies over the contra ESG counterparts. Table 4 complements table 3 with a breakdown of significance for each coefficient in table 3. The low p-values in the EMP factor indicate that the coefficients on all but the largest and smallest size deciles are significant at the 5\% level. It is important to note that all of the other Fama French factors are also less significant In the largest and small portfolios. This trend might indicate that the data might have pointed at far ends which do not behave like the other deciles.  Table 6 demonstrates the risk premia for each of the factors using the second step in the two-pass sort. All of the Fama French factors in the size decile two-pass sort have positive risk premia. The EMP factor large than all of the other factors, with a coefficient of approximately  16.8\%




Table 5 details the relationship outlined in equation \eqref{six_factor_basic_reg} in the methodology where the column axis is the beta decile. In this table,  the beta decile 1 comprises the companies with the lowest beta. Similarly, beta decile 10 is the portfolio of companies with the largest beta values.  Similar to the size decile, the EMP factor exhibits a negative relation with regard to its sensitivity. However, unlike the size decile analysis, there does not appear to be a significant relationship between the beta decile and the factor's sensitivity. Although the most significant impact is on the lowest decile, the least impact is experienced on the 6th decile portfolio, and the largest decile is close to the middle in terms of sensitivity. Similar to the size decile analysis, the EMP coefficients are all very significant, with every decile having significance at the 5\% level. When considering the second phase of the two-pass sort, the similarities to the size decile continue wherein each of the Fama French five factors exhibit positive risk premia, and the EMP also exhibits a positive relationship. The EMP factor again has a high coefficient with approximately 15.8\%. 
